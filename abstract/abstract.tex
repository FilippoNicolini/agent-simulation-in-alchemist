%\pagenumbering{arabic} % serve per mettere i numeri romani
\chapter*{\hl{Abstract}} % crea l'introduzione (un capitolo non numerato)

% imposta l'intestazione di pagina
\rhead[\fancyplain{}{\bfseries ABSTRACT}]{\fancyplain{}{\bfseries\thepage}}
\lhead[\fancyplain{}{\bfseries\thepage}]{\fancyplain{}{\bfseries ABSTRACT}}

% aggiunge la voce Introduzione nell'indice
\addcontentsline{toc}{chapter}{Abstract}

In questo lavoro di tesi si vuole presentare un nuovo linguaggio che fonda le sue basi su AgentSpeak e tuProlog permettendo quindi di implementare il modello ad agenti, ereditato da AgentSpeak, su ambienti o piattaforme diverse, grazie alla flessibilità di tuProlog. Verrà preso in esame l'utilizzo del linguaggio in combinazione con il meta-simulatore Alchemist e ne verranno descritti i dettagli implementativi.

% non numera l'ultima pagina sinistra
\clearpage{\pagestyle{empty}\cleardoublepage}

%crea l'indice
\tableofcontents

% imposta l'intestazione di pagina
\rhead[\fancyplain{}{\bfseries\leftmark}]{\fancyplain{}{\bfseries\thepage}}
\lhead[\fancyplain{}{\bfseries\thepage}]{\fancyplain{}{\bfseries INDICE}}

% non numera l'ultima pagina sinistra
\clearpage{\pagestyle{empty}\cleardoublepage}

% crea l'elenco delle figure
\listoffigures

% non numera l'ultima pagina sinistra
\clearpage{\pagestyle{empty}\cleardoublepage}

% crea l'elenco delle tabelle
%\listoftables

% non numera l'ultima pagina sinistra
\clearpage{\pagestyle{empty}\cleardoublepage}

% crea l'elenco dei codici sorgenti
\lstlistoflistings

% non numera l'ultima pagina sinistra
\clearpage{\pagestyle{empty}\cleardoublepage}