%\pagenumbering{arabic} % serve per mettere i numeri romani
\chapter*{Abstract} % crea l'introduzione (un capitolo non numerato)

% imposta l'intestazione di pagina
\rhead[\fancyplain{}{\bfseries ABSTRACT}]{\fancyplain{}{\bfseries\thepage}}
\lhead[\fancyplain{}{\bfseries\thepage}]{\fancyplain{}{\bfseries ABSTRACT}}

% aggiunge la voce Introduzione nell'indice
\addcontentsline{toc}{chapter}{Abstract}

L'obiettivo del lavoro di tesi è quello di realizzare un linguaggio che permetta di implementare il modello ad agenti BDI su ambienti e piattaforme differenti, ovvero dove l'ambiente di esecuzione può essere sia quello simulato che quello reale (ad esempio la JVM). Si vuole quindi realizzare un ambiente di lavoro che permetta di eseguire il modello ad agenti BDI, definito da AgentSpeak e implementato in Jason, attraverso l'utilizzo del meta-simulatore Alchemist, definendo la relativa incarnazione, e sfruttando tuProlog sia per la gestione del linguaggio che per quella dello stato e dei piani dell'agente.

% non numera l'ultima pagina sinistra
\clearpage{\pagestyle{empty}\cleardoublepage}

%crea l'indice
\tableofcontents

% imposta l'intestazione di pagina
\rhead[\fancyplain{}{\bfseries\leftmark}]{\fancyplain{}{\bfseries\thepage}}
\lhead[\fancyplain{}{\bfseries\thepage}]{\fancyplain{}{\bfseries INDICE}}

% non numera l'ultima pagina sinistra
\clearpage{\pagestyle{empty}\cleardoublepage}

% crea l'elenco delle figure
\listoffigures

% non numera l'ultima pagina sinistra
\clearpage{\pagestyle{empty}\cleardoublepage}

% crea l'elenco delle tabelle
%\listoftables

% non numera l'ultima pagina sinistra
\clearpage{\pagestyle{empty}\cleardoublepage}

% crea l'elenco dei codici sorgenti
\lstlistoflistings

% non numera l'ultima pagina sinistra
\clearpage{\pagestyle{empty}\cleardoublepage}