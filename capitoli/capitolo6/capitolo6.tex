% crea il CAPITOLO
\chapter*{Conclusioni}
% imposta l'intestazione di pagina
\lhead[\fancyplain{}{\bfseries\thepage}]{\fancyplain{}{\bfseries\rightmark}}
% mette i numeri arabi
%\pagenumbering{arabic}

Il tempo speso per questa fase di preparazione alla prova finale è stato molto utile poichè ha permesso di conoscere in modo approfondito gli strumenti e di padroneggiarli. In questo documento è stata descritta un'evoluzione a partire dai problemi iniziali riscontrati con l'approccio a nuovi strumenti di sviluppo, passando poi per una fase di analisi e studio che, infine, ha portato alla risoluzione di semplici problemi (come lo scambio di messaggi o lo spostamento di un nodo), i quali hanno consentito di prendere ancora più famigliarità con Alchemist, tuProlog e il modello ad agenti.
Le problematiche affrontate non sono state scelte a caso ma selezionate pensando a situazioni precise che si sarebbero dovute affrontare per la realizzazione del progetto per la tesi.

Vi sono ancora problemi aperti, soprattutto legati alle integrazioni, quali:
\begin{itemize}
\item organizzazione struttura classi
\item la gestione del ciclio di ragionamento dell'agente
\item utilizzo risorse tuProlog come teoria dell'agente
\end{itemize}
che saranno affrontati nella prossima fase di analisi, durante il design, prima di iniziare l'implementazione dell'interprete.

Alcune delle parti sviluppate nella risoluzione dei sotto-problemi saranno prese come spunto per ottimizzazioni e la realizzazione del progetto di tesi.