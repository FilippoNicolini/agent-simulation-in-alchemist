\chapter{Conclusioni}\label{chap:conclusions}
Con questo lavoro di tesi è stato definito un interprete Prolog con delle API per la programmazione ad agenti basato sulla struttura del modello di AgentSpeak al quale è stato integrata la gestione del ragionamento degli agenti proposta dall'interprete Jason.
%
Il linguaggio definito permette di scrivere teorie Prolog per implementare il comportamento degli agenti sfruttando gli eventi e i predicati messi a disposizione.
Questo linguaggio non si occupa di gestire la piattaforma sulla quale opereranno gli agenti ma solo di dare una definizione degli elementi che permetteranno all'agente di essere autonomo, reattivo e proattivo.

Questo linguaggio potrà essere implementato in qualsiasi ambiente, da quello reale a quello simulato, poichè grazie all'utilizzo della libreria tuProlog sarà possibile realizzare efficientemente la gestione delle interazioni con le primitive.
%
Per l'interprete del linguaggio sarà necessario implementare un applicativo che gestisca lo scheduling degli eventi, in caso di ambiente simulato, o la comunicazione con i percettori, in caso di ambiente reali, e trasmetta le informazioni alla teoria dell'agente utilizzando il linguaggio attraverso la libreria tuProlog.

All'interno del lavoro di tesi è descritto come è stata realizzata un'incarnazione utilizzando il simulatore Alchemist.
Con questo esempio, anche attraverso alla flessibilità del meta-modello di Alchemist, si è mostrata la grande capacità di adattamento del linguaggio e la trasparenza con cui l'interprete può gestire gli eventi e gli agenti.

\section{Vantaggi approccio scelto}
Il linguaggio è stato realizzato con l'obiettivo di fornire una nuova prospettiva più snella e dinamica che permetta di portare la potenza del modello ad agenti su varie piattaforme cercando di semplificare il lavoro di integrazione e di implementazione con questo modello.

Attraverso il linguaggio definito in questo lavoro è possibile portare il modello ad agenti da un ambiente simulato, come quello che è stato realizzato all'interno di questo progetto di tesi, ad un ambiente reale, realizzando ad esempio un applicativo per dispositivi mobile ognuno dei quali esegue degli agenti, i quali comunicano con altri agenti presenti in differenti device.

Un grande vantaggio della soluzione realizzata in questo progetto è il fatto di aver utilizzato compoenti, quali Alchemist e tuProlog, solide e supportate da una community che ne garantisce la stabilità e l'aggiornamento nel tempo.
Grazie a queste componenti la manutenibilità di quanto prodotto risulterà meno onersa e permetterà di concentrarsi sull'implementazione di migliorie.
