% crea il CAPITOLO
\chapter{tuProlog}
% imposta l'intestazione di pagina
\lhead[\fancyplain{}{\bfseries\thepage}]{\fancyplain{}{\bfseries\rightmark}}
% mette i numeri arabi
%\pagenumbering{arabic}

tuProlog è stato scelto per essere utilizzato nel progetto di tesi per via della sua capacità di operare su ambienti differenti. Grazie a questo strumento e alla sua flessibilità, sarà possibile integrare le due parti, l'interprete e il linguaggio, su uno dei possibili ambienti di sviluppo, ovvero Alchemist. Infatti, proprio grazie al fatto che la libreria tuProlog supporta la programmazione multi-paradigma, è possibile utilizzare vari ambienti per la realizzazione di un'interprete che sfrutti il linguaggio che si andrà a definire nel lavoro di tesi.

%----------------------------
\section{Descrizione tuProlog}
tuProlog è un interprete Prolog per le applicazioni e le infrastrutture Internet basato su Java. \`E progettato per essere facilmente utilizzabile, leggero, configurabile dinamicamente, direttamente integrato in Java e facilmente interoperabile.
tuProlog è sviluppato e mantenuto da `aliCE' un gruppo di ricerca dell'Alma Mater Studiorum - Università di Bologna, sede di Cesena. \`E un software Open Source e rilasciato sotto licenza LGPL.

\subsection{Caratteristiche tuProlog}
tuProlog è un sistema per applicazioni e infrastrutture distribuite ed è intenzionalmente strutturato con un core minimale, per essere configurato sia staticamente che dinamicamente tramite l'utilizzo di librerie e predicati \cite{tuProlog}. Inoltre, tuProlog supporta nativamente la programmazione multi-paradigma, fornendo un'integrazione chiara ed efficiente tra Prolog e i principali linguaggi orientati agli oggetti, ad esempio Java.

tuProlog ha diverse caratteristiche e qui di seguito verranno illustrate solo alcune di esse, ovvero quelle utilizzate all'interno di questo lavoro.
Il motore tuProlog fornisce e riconosce i seguenti tipi di predicati:
\begin{itemize}
\item predicati built-in: incapsulati nel motore tuProlog;
\item predicati di libreria: inseriti in una libreria che viene caricata nel motore tuProlog. La libreria può essere liberamente aggiunta all'inizio o rimossa dinamicamente durante l'esecuzione. I predicati della libreria possono essere sovrascritti da quelli della teoria. Per rimuovere un singolo predicato dal motore è necesssario rimuovere tutta la libreria che contiene quel predicato;
\item predicati della teoria: inseriti in una teoria che viene caricata nel motore tuProlog. Le teorie tuProlog sono semplicemente collezioni di clausole Prolog. Le teorie possono essere liberamente aggiunte all'inizio o rimosse dinamicamente durante l'esecuzione.
\end{itemize}

\subsubsection{Prolog}
Prolog è un linguaggio di programmazione che adotta il paradigma della programmazione logica. \`E impiegato in molti programmi di intelligenza artificiale.

Prolog si basa sul calcolo dei predicati (più precisamente di quelli del primo ordine). L'esecuzione di un programma Prolog è comparabile alla dimostrazione di un teorema mediante regola di inferenza. I concetti fondamentali sono l'unificazione, la ricorsione in coda e il backtracking, o monitoraggio a ritroso.

\section{Aspetti principali in tuProlog}
tuProlog, come detto precedentemente, è un interprete semplificato di Prolog.
Le difficoltà iniziali nell'utilizzo di questo strumento erano dovute alla carenza di conoscenza di Prolog e della libreria in generale.

tuProlog è uno strumento molto utile perchè riesce ad adattasi all'ambiente sul quale viene utilizzato. In questo caso, per il progetto di tesi, è stato scelto Alchemist e l'adattabilità di tuProlog si è rivelata davvero efficiente poichè in poco tempo si è riusciti ad implementare la gestione della teoria dell'agente. 

tuProlog è uno strumento che ha la possibilità di essere utilizzato in maniera differente in base all'ambito applicativo, poichè mette a disposizione una serie di predicati nativi (o built-in) oltre che delle librerie per la gestione dei vari paradigmi supportati.
\\
Una peculiare modalità di utilizzo di tuProlog, fornita per il paradigma orientato agli oggetti,  è la registrazione di istanze di classi all'interno di varibili inserite nella teoria. In questo modo è possibile modificare le proprietà dell'istanza o invocare i suoi metodi utilizzando la variabile tuProlog direttamente dalla teoria.
\\
Questa funzionalità verrà utilizzata nello sviluppo del progetto di tesi per integrare l'ambiente dell'interprete, che sarà Alchemist, con l'agente e il linguaggio che verrà definito nel lavoro di tesi.