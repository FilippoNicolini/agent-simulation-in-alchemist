% crea il CAPITOLO
\chapter{Introduzione}
% imposta l'intestazione di pagina
\lhead[\fancyplain{}{\bfseries\thepage}]{\fancyplain{}{\bfseries\rightmark}}
In questo lavoro preparatorio vengono descritti i modelli e gli strumenti che saranno poi utilizzati nel progetto di tesi (Alchemist, tuProlog, modello agenti BDI AgentSpeak) mostrando come sono composti e quali sono i loro punti salienti. Successivamente viene descritto il lavoro che è stato fatto su ogni ambiente per conoscere sia Alchemist che tuProlog, in modo da ottenere informazioni utili per impostare in modo corretto il progetto di tesi.

\section{Obiettivo della tesi}
L'obiettivo del lavoro di tesi è quello utilizzare la definizione di agenti BDI fatta da AgentSpeak per definire un nuovo linguaggio ad agenti al quale, inoltre, si è voluto aggiungere anche una caratteristica di flessibilità. Quest'ultima è stata raggiunta grazie all'utilizzo della libreria tuProlog che ha permesso di definire un linguaggio che possa essere utilizzato da interpreti realizzati su ambienti e piattaforme differenti accomunate dall'utilizzo di questa libreria.
Si vuole mostrare, inoltre, come è possibile creare un interprete all'interno del meta-simulatore Alchemist, realizzando un'opportuna incarnazione, che sfrutta il modello di agenti BDI definito da AgentSpeak e l'implementazione di Jason, in particolare relativamente al ciclo di ragionamento, e che permette l'esecuzione di agenti, definiti tramite le teorie utilizzando il nuovo linguaggio, in un ambiente simulato.

\section{Obiettivo attività propedeutica}
L'attività descritta in questo documento ha quindi l'obiettivo di analizzare e studiare Alchemist, tuProlog e il modello ad agenti BDI di AgentSpeak recependo le informazioni necessarie per permettere di impostare il lavoro di tesi in maniera più efficace.

Per quanto riguarda Alchemist l'obiettivo è quello di imparare ad utilizzarlo, studiando: la struttura del suo meta-modello; come scrivere la configurazione della simulazione; come implementare un'incarnazione.

Anche con tuProlog l'obiettivo è di studiare le sue funzionalità e il suo utilizzo, anche attraverso la comprensione di Prolog. Il processo di acquisizione di informazioni, descritto nel documento, racconta dello studio iniziato su Prolog, per comprendere le nozioni di base, e poi proseguito sulla libreria tuProlog, attraverso il manuale.

In questa attività si è inoltre analizzato il modello ad agenti BDI proposto da AgentSpeak, sia per quanto riguarda la composizione che per il ciclo di ragionamento; per quest'ultimo è stato preso come esempio il processo definito in Jason, una delle implementazioni già note di AgentSpeak. In particolare, si è dedicato tempo all'approfondimento del ciclo di ragionamento per captarne l'utilizzo degli elementi e poter successivamente, per il progetto di tesi, riuscire a strutturare opportunamente l'interprete realizzato con Alchemist.

Si è così potuto fare un raffronto tra il modello ad agenti e il meta-modello di Alchemist, cercando i possibili modi di effettuare l'integrazione tra gli agenti e Alchemist, utilizzando come collante tuProlog.
