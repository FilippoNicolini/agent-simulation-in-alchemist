% crea il CAPITOLO
\chapter{Introduzione}
% imposta l'intestazione di pagina
\lhead[\fancyplain{}{\bfseries\thepage}]{\fancyplain{}{\bfseries\rightmark}}
TODO DESCRIZIONE INTRODUZIONE

\section{Obiettivo del lavoro}
TODO 
%Le tecnologie appena descritte permettono di programmare agenti, ognuna con funzionalit\`a specifiche come la distribuzione, protocolli di comunicazione e il modello BDI.
%Nessuna di quelle descritte per\`o \`e in grado di fornire la possibilit\`a di tener conto della distribuzione degli agenti nel mondo reale.

%Con questo lavoro di tesi si propone un linguaggio ed un'architettura ad agenti che possa essere eseguita sia in un ambiente simulato che in uno reale, per tentare di esprimere tutto il potenziale del modello ad agenti.
%In particolare, il focus della tesi \`e definire e formalizzare il linguaggio e poi effettuare l'esecuzione in un ambiene simulato.
%L'applicativo utilizzato per la simulazione \`e il meta-simulatore Alchemist, che fornisce un ambiente sul quale \`e possibile sviluppare definizioni di modelli da eseguire su di esso.

\subsection{Benefici dell'approccio scelto}
TODO 