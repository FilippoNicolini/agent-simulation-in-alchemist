% crea il CAPITOLO
\chapter{Introduzione}
% imposta l'intestazione di pagina
\lhead[\fancyplain{}{\bfseries\thepage}]{\fancyplain{}{\bfseries\rightmark}}
TODO DESCRIZIONE INTRODUZIONE

% crea la SEZIONE
\section{Contesto}
TODO DESCRIZIONE CONTESTO

\subsection{JADE}
JADE (Java Agent DEvelopement Framework) \`e un software implementato in Java che semplifica l'implementazione del sistema multi-agente attraverso un middleware che \`e conforme alle specifiche FIPA e uno strumento grafico che supporta le fasi di debug e distribuzione.

FIPA \`e una societ\`a di standard IEEE, il cui scopo \`e la promozione di tecnologie e specifiche di interoperabilit\`a che facilitino l'interworking end-to-end di sistemi di agenti intelligenti in moderni ambienti commerciali ed industriali.

Un middleware \`e un software che fornisce servizi per applicazioni che permette agli sviluppatori di implementare meccanismi di comunicazione e di input/output. Viene usato particolarmente in software che necessitano di comunicazione e gestione di dati in applicazioni distribuite.

Un sistema basato su JADE pu\`o essere distribuito su diverse macchine (anche con sistemi operativi differenti) e la configurazione pu\`o essere controllata da un'interfaccia remota. La configurazione pu\`o essere anche cambiata durante l'esecuzione spostando gli agenti da una macchina ad un'altra in base alle necessit\`a.

%Dietro alla parte di astrazione degli agenti, JADE fornisce un semplice e potente modello di esecuzione e composizione dei task, la comunicazione peer to peer tra gli agenti basata sul paradigma di scambio di messaggi asincrono.

L'architettura di comunicazione offre uno scambio di messaggi privati di ogni agente  flessibile ed efficiente, dove JADE crea e gestisce una coda di messaggi ACL in entrata.
\`E stato implementato il modelllo FIPA completo e i suoi componenti sono stati ben distinti e pienamente integrati: interazione, protocolli, preparazione pacchetti, ACL, contenuto dei linguaggi, schemi di codifica, ontologie e protocollo di trasporto.
Il meccaniscmo di trasporto, in particolare, \`e come un camaleonte perch\`e si adatta ad ogni situazione scegliendo trasparentemente il miglior protocollo disponibile.

\subsection{SPADE}
SPADE (Smart Python multi-Agent Development Environment) \`e una piattaforma per sistemi multi-agente scritta in Python e basata sui messaggi istantanei (XMPP).
Il protocollo XMPP offre una buon architettura per la comunicazione tra agenti in modo strutturato e risolve eventuali problemi legati al design della piattaforma, come autenticazione degli utenti (agenti) o creazione di canali di comunicazione.

Il modello ad agenti \`e composto da un meccanismo di connessione alla piattaforma, un dispatcher di messaggi e un set di comportamenti differenti a cui il dispatcher da i messaggi. Ogni agente ha un identificativo (JID) e una password per autenticarsi al server XMPP.

La connessione alla piattaforma \`e gestita internamente tramite il protocollo XMPP, il quale fonisce un meccaniscmo per registrare e autenticare gli utenti al server XMPP. Ogni agente potr\`a quindi mantenere aperta e persistente uno stream di comunicazioni con la piattaforma.

Ogni agente ha al suo interno un componente dispatcher per i messaggi che opera come un postino: quando arriva un messaggio per l'agente, lo posizione nella corretta casella di posta; quando l'agente deve inviare un messaggio, il dispatcher si occupa di inserirlo nello stream di comunicazione.

Un agente pu\`o avere pi\`u comportamenti simultaneamente. Un comportamento \`e un operazione che l'agente pu\`o eseguire usando il pattern di ripetizione. Spade fornisce alcuni comportamenti predefiniti: Cyclic, Periodic (utili per eseguire operazioni ripetitive); One-Shot, Time-Out (usati per eseguire operazioni casuali); Finite State Machine (permette di costruire comportamenti complessi).
Quando un messaggio arriva all'agente, il dispatcher lo indirizza alla coda del comportamento corretto: il dispatcher utilizza il template di messaggi di ogni comportamento per capire qual \`e il giusto destinatario. Quindi un comportamento pu\`o definire il tipo di messaggi che vuole ricevere.

\subsection{Jason}
Jason \`e un interprete per la versione estesa di AgentSpeak che implementa la semantica operazionale di tale linguaggio e fornisce una piattaforma per lo sviluppo di sistemi multi-agente. AgentSpeak \`e uno dei principali linguaggi orientati agli agenti basati sull'architettura BDI. Il linguaggio interpretato da Jason \`e un'estensione del linguaggio di programmazione astratto AgentSpeak(L).
Gli agenti BDI (Belief-Desires-Intentions) forniscono un meccaniscmo per separare le attivit\`a di selezione di un piano, fra quelli disponibili, dall'esecuzione del piano attivo, permettendo di bilanciare il tempo per la scelta del piano e quello per eseguirlo.

\subsection{SARL}
SARL \`e un linguaggio di programmazione ad agenti tipizzato staticamente. SARL mira a fornire le astrazioni fondamentali per affrontare la concorrenza, la distribuzione, l'interazione, il decentramento, la reattivit\`a e la riconfigurazione dinamica. Queste funzionalit\`a di alto livello sono adesso considerate i principali requisiti per un'implementazione facile e pratica delle moderne applicazioni software complesse.

\subsection{JADEX}
Il framework di componenti attivi Jadex fornisce funzionalit\`a di programmazione e di esecuzione per sistemi distribuiti e concorrenti. L'idea generale \`e di considerare che il sistema sia composto da componenti che agiscono come fornitori di serviziw e consumatori.
Rispetto a SCA (Service Component Architecture) i componenti sono sempre entit\`a attive, mentre in confronto agli agenti la comunicazione \`e preferibilmente eseguita utilizzando chiamate ai servizi.

\subsection{ASTRA}
ASTRA \`e un linguaggio di programmazione ad agenti per creare sistemi intelligenti distrubuiti/concorrenti costruiti su Java.
ASTRA \`e basato su AgentSpeak(L), ovvero fornisce tutte le stesse funzionalit\`a base, ed inoltre le aumenta con una serie di feature orientate a creare un linguaggio di programmazione ad agenti pi\`u pratico.


%--------------------------------------
%Le caratteristiche di un agente sono:
%\begin{itemize}
%	\item Autonomia
%	\item Reattivit\`a
%	\item Proattivit\`a
%	\item Orientamento ai goal
%	\item Socialit\`a
%	\item Adatattivit\`a
%	\item Cognitivit\`a
%\end{itemize}


\section{Obiettivo del lavoro}
TODO 
%Le tecnologie appena descritte permettono di programmare agenti, ognuna con funzionalit\`a specifiche come la distribuzione, protocolli di comunicazione e il modello BDI.
%Nessuna di quelle descritte per\`o \`e in grado di fornire la possibilit\`a di tener conto della distribuzione degli agenti nel mondo reale.

%Con questo lavoro di tesi si propone un linguaggio ed un'architettura ad agenti che possa essere eseguita sia in un ambiente simulato che in uno reale, per tentare di esprimere tutto il potenziale del modello ad agenti.
%In particolare, il focus della tesi \`e definire e formalizzare il linguaggio e poi effettuare l'esecuzione in un ambiene simulato.
%L'applicativo utilizzato per la simulazione \`e il meta-simulatore Alchemist, che fornisce un ambiente sul quale \`e possibile sviluppare definizioni di modelli da eseguire su di esso.

\subsection{Benefici dell'approccio scelto}
TODO 