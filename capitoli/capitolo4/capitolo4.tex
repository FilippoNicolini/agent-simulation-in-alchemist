% crea il CAPITOLO
\chapter{Agent speak}
% imposta l'intestazione di pagina
\lhead[\fancyplain{}{\bfseries\thepage}]{\fancyplain{}{\bfseries\rightmark}}
% mette i numeri arabi
%\pagenumbering{arabic}

AgentSpeak è un linguaggio orientato agli agenti basato sulla programmazione logica e sul modello BDI (Belief-Desires-Intention). All'interno del capitolo viene descritto cos'è il modello BDI e come funziona AgentSpeak, ed in particolare il ciclo di ragionamento, esemplificato da quello di Jason.

%----------------------------
\section{Agenti BDI con AgentSpeak}
Il modello BDI consente di rappresentare le caratteristiche e le modalità di raggiungimento di un obiettivo secondo il paradigma ad agenti. Gli agenti BDI forniscono un meccanismo per separare le attività di selezione di un piano, fra quelli presenti nella sua teoria, dall'esecuzione del piano attivo, permettendo di bilanciare il tempo speso nella scelta del piano e quello per eseguirlo.

I \textbf{beliefs} sono quindi informazioni dello stato dell'agente, ovvero ciò che l'agente sa del mondo \cite{AgentSpeakInJason} il suo insieme è chiamato `belief base' o `belief set'.

I \textbf{desires} rappresentano tutti i possibili piani che un agente potrebbe eseguire \cite{AgentSpeakInJason}. Rappresentano ciò che l'agente vorrebbe realizzare o portare a termine: i \textit{goals} sono desideri che l'agente persegue attivamente ed è quindi bene che tra loro siano coerenti, cosa che non è obbligatoria per quanto riguarda il resto dei desideri.

Le \textbf{intentions} identificano i piani a cui l'agente ha deciso di lavorare o a cui sta già lavorando e a loro volta possono contenere altri piani \cite{AgentSpeakInJason}.

Gli \textbf{eventi} innescano le attività reattive, ovvero la caratteristica di proattività degli agenti, come ad esempio l'aggiornamento dei beliefs, l'invocazione di piani o la modifica dei goals.

\section{Definizione AgentSpeak}
\textbf{AgentSpeak} è un linguaggio di programmazione basato su un linguaggio del primo ordine con eventi e azioni \cite{AgentSpeak(L)}. Il comportamento degli agenti è dettato da quanto definito nel programma scritto in AgentSpeak. I beliefs correnti di un agente sono relativi al suo stato attuale, all'enviroment e agli altri agenti. Gli stati che un agente vuole determinare sulla base dei suoi stimoli esterni e interni sono i desideri \cite{AgentSpeak(L)}. L'adozione di programmi per soddisfare tali stimoli è detta intenzioni.

\subsection{Composizione}
Un'agente in AgentSpeak è formato da una `belief base' e da una serie di piani opportunamente programmati.
La `belief base' è il contenitore dello stato dell'agente, dove sono presenti tutte le informazioni che esso ha in riferimento a se stesso e all'ambiente. Questo set di belief può essere modificato in modo continuo dalle azioni scatenate nel ciclo di ragionamento.
I piani sono sequenze di azioni o goal che permettono all'agente di reagire a situazioni che avvengono nell'ambiente o internamente.
\\
Qui di seguito viene descritto quali sono le fasi del ciclo di ragionamento dell'agente e in che modo viene definito il suo operato. La sequenza di esecuzione di questa iterazione è fondamentale per la realizzazione di un agente e quindi importante comprenderla per realizzare un'architettura appropriata.

\subsection{Ciclo di ragionamento}\label{ssctn:cicloRagionamentoAgentSpeak}
Il ciclo di ragionamento è il modo in cui l'agente prende le sue decisioni e mette in pratica le azioni. Esso è composto di otto fasi: le prime tre sono quelle che riguardano l'aggiornamento dei belief relativi al mondo e agli altri agenti, mentre altre descrivono la selezione di un evento che permette l'esecuzione di un'intenzione dell'agente.

%/---AGGIORNAMENTO BELIEF BASE---/
\subsubsection{a. Percezione ambiente}
La percezione effettuata dall'agente all'interno del ciclo di ragionamento è utilizzata per poter aggiornare il suo stato. L'agente interroga dei componenti capaci di rilevare i cambiamenti nell'ambiente \cite{AgentSpeakInJason} e di emettere dati consultabili utilizzando opportune interfacce.

\subsubsection{b. Aggiornamento beliefs}
Ottenuta la lista delle percezioni è necessario aggiornare la `belief base'. Ogni percezione non ancora presente nel set viene aggiunta e al contrario quelle presenti nel set e che non sono nella lista delle percezioni vengono rimosse \cite{AgentSpeakInJason}.
Ogni cambiamento effettato nella `belief base' produce un evento: quelli generati da percezioni dell'ambiente sono detti eventi esterni; quelli interni, rispetto agli altri, hanno associata un'intenzione.

\subsubsection{c. Ricezione e selezione messaggi}
L'altra sorgente di informazioni per un agente sono gli altri agenti presenti nel sistema. L'interprete controlla i messaggi diretti all'agente e li rende a lui disponibili \cite{AgentSpeakInJason}: ad ogni iterazione del ciclo può essere processato solo un messaggio. Inoltre,può essere assegnata una priotità ai messaggi in coda definendo una funzione di prelazione per l'agente.
\\
Prima di essere processati i messaggi passano all'interno di una funzione di selezione che definisce quali messaggi possano essere accettati dall'agente \cite{AgentSpeakInJason}. Questa funzione può essere implementata ad esempio per far ricevere solo i messaggi di un certo agente.

%/---SELEZIONE EVENTO E ESECUZIONE INTENZIONE---/
\subsubsection{d. Selezione evento}
Gli eventi rappresentano la percezione del cambiamento nell'ambiente o dello stato interno dell'agente \cite{AgentSpeakInJason}, come il goal. Ci possono essere vari eventi in attesa ma in ogni ciclo di ragionamento può esserne gestito uno solo, il quale viene scelto dalla funzione di selezione degli eventi che ne seleziona uno dalla lista di quelli in attesa. Se la lista di eventi fosse vuota si passa direttamente alla penultima fase del ciclo di ragionamento \cite{AgentSpeakInJason}, ovvero la selezione di un'intenzione.

\subsubsection{e. Recupero piani rilevanti}
Una volta selezionato l'evento è necessario trovare un piano che permetta all'agente di agire per gestirlo. Per fare ciò viene recuperata dalla `Plan Library' la lista dei piani rilevanti, verificando quali possano essere unificati con l'evento selezionato \cite{AgentSpeakInJason}. L'unificazione è il confronto relativo a predicati e termini. Al termine di questa fase si otterrà un set di piani rilevanti per l'evento selezionato che verrà raffinato successivamente.

\subsubsection{f. Selezione piano appplicabile}
Ogni piano ha un contesto che definisce con quali informazioni dell'agente può essere usato.
Per piano applicabile si intendeno quelli che, in relazione allo stato dell'agente, possono avere una possibilità di successo. Viene quindi controllato che il contesto sia una conseguenza logica della `belief base' dell'agente \cite{AgentSpeakInJason}. Vi possono anche essere più piani in grado di gestire un evento ma l'agente deve selezionarne uno solo ed impegnarsi ad eseguirlo.
\\
La selezione viene fatta tramite un'apposita funzione che inoltre tiene conto dell'ordinamento dei piani in base alla loro posizione nel codice sorgente oppure dell'ordine di inserimento. Quando un piano è scelto, viene creata un'istanza di quel piano che viene inserita nel set delle intenzioni \cite{AgentSpeakInJason}: sarà l'istanza ad essere manipolata dall'interprete e non il piano nella libreria.

Ci sono due possibili modalità per la creazione di un'intenzione e dipende dal fatto che l'evento selezionato sia esterno o interno \cite{AgentSpeakInJason}. Nel primo caso viene semplicemente creata l'intenzione, altrimenti viene inserita un'altra intenzione in testa a quella che ha generato l'evento, poichè è necessario eseguire fino al completamento un piano per raggiungere tale goal.

\subsubsection{g. Selezione intenzione}
A questo punto, se erano presenti eventi da gestire, è stata aggiunta un'altra intenzione nello stack. Un agente ha tipicamente più di un'intenzione nel set delle intenzioni che potrebbe essere eseguita, ognuna delle quali rappresenta un diverso punto di attenzione \cite{AgentSpeakInJason}. Ad ogni ciclo di ragionamento avviene l'esecuzione di una sola intenzione, la cui scelta è importante per come l'agente opererà nell'ambiente.

\subsubsection{h. Esecuzione intenzione}
L'intenzione, scelta nello step precedente, non è altro che il corpo di un piano formato da una sequenza di istruzioni, ognuna delle quali, una volta eseguita, viene rimossa dall'istanza del piano. Terminata l'esecuzione un'intenzione, quest'ultima viene restituita al set delle intenzioni a meno che non debba aspettare un messaggio o un feedback dell'azione \cite{AgentSpeakInJason}: in questo caso viene memorizzata in una struttura e restituita una volta ricevuta la risposta.
Se un'intenzione è sospesa non può essere selezionata per l'esecuzione nel ciclo di ragionamento.

\subsubsection{Scambio di messaggi}
Lo scambio di messaggi è la comunicazione standard che avviene tra agenti per comunicare tra loro e operare in base al contenuto ricevuto.
La comunicazione definita da AgentSpeak utilizza tre parti. La prima è la coda dei messaggi in input, ovvero una lista contenente tutti i messaggi che il sistema o interprete riceve e che sono destinati all'agente. La seconda è la coda dei messaggi di output che si allunga ogni volta che l'agente vuole inviare un messaggio ad un altro agente. L'ultima è una struttura all'interno della quale vengono memorizzate le intenzioni che sono sospese dall'esecuzione poichè aspettano una risposta dal canale di comunicazione dei messaggi.
\\
L'interprete è il mezzo per il quale i messaggi trasmessi. Esso infatti ha il compito di recuperare tutti i messaggi nella coda in uscita di ogni agente e successivamente recapitarli. Per la consegna viene recuperato l'agente destinatario di ogni messaggio e poi quest'ultimo viene posizionato nella coda di quelli in input dell'agente, in modo tale che possa recuperarne il contenuto al prossimo ciclo di ragionamento.


%----------------------------

