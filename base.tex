% 12pt: grandezza carattere
% a4paper: formato a4
% openright: apre i capitoli a destra
% twoside: serve per fare un
% documento fronteretro
% report: stile tesi (oppure book)
\documentclass[12pt,a4paper,openright,twoside]{book}

% libreria per accettare i caratteri
% digitati da tastiera come ?
% si può usare anche
% \usepackage[T1]{fontenc}
% però con questa libreria
% il tempo di compilazione
% aumenta
\usepackage[utf8]{inputenc}

%libreria per scrivere in italiano
\usepackage[italian]{babel}

\usepackage{hyperref}

% libreria per impostare il documento
\usepackage{fancyhdr}

% libreria per avere l'indentazione all'inizio dei capitoli, ...
\usepackage{indentfirst}

%libreria per mostrare le etichette
%\usepackage{showkeys}

% libreria per inserire grafici
\usepackage{graphicx}

% libreria per utilizzare font particolari ad esempio \textsc{}
\usepackage{newlfont}

% librerie matematiche
\usepackage{amssymb}
\usepackage{amsmath}
\usepackage{latexsym}
\usepackage{amsthm}
\usepackage[italian]{cleveref}
\usepackage{geometry}

\theoremstyle{plain}
\newtheorem{thm}{Theorem}[chapter] % reset theorem numbering for each chapter

\theoremstyle{definition}
\newtheorem{defn}[thm]{Definition} % definition numbers are dependent on theorem numbers
\newtheorem{exmp}[thm]{Example} % same for example numbers

%per inserire il codice
\usepackage{listings, xcolor}
\renewcommand{\lstlistingname}{Codice sorgente}
\renewcommand{\lstlistlistingname}{Codici sorgenti}

%per l'highlight delle modifiche
\usepackage{color,soul}

%URL
\usepackage{url}

%note a piè di pagina che non riniziano da 1 ad ogni capitolo
\usepackage{chngcntr}
\counterwithout{footnote}{chapter}

% impostano i margini
\oddsidemargin=30pt
\evensidemargin=20pt

% serve per la sillabazione: tra parentesi vanno inserite come nell'esempio le parole
% che latex non riesce a tagliare nel modo giusto andando a capo.
\hyphenation{sil-la-ba-zio-ne pa-ren-te-si}

% comandi per l'impostazione della pagina, vedi il manuale della libreria fancyhdr per ulteriori delucidazioni
\pagestyle{fancy}\addtolength{\headwidth}{20pt}
\renewcommand{\chaptermark}[1]{\markboth{\thechapter.\ #1}{}}
\renewcommand{\sectionmark}[1]{\markright{\thesection \ #1}{}}
\rhead[\fancyplain{}{\bfseries\leftmark}]{\fancyplain{}{\bfseries\thepage}}
\cfoot{}


\newcommand{\impliedBy}{\ \text{\texttt{:-}}\ }
\newcommand{\leftArrow}{\text{\texttt{`<-'}}}

\newcommand{\switchToJava}[1]{
\lstset{
	numberstyle=\footnotesize\color{black},
	basicstyle=\ttfamily#1,
	breakatwhitespace=true,
	breaklines=true,
	captionpos=b,
	keepspaces=true,
	numbers=left,
	numbersep=7pt,
	showspaces=false,
	showstringspaces=false,
	showtabs=false,
	tabsize=2,
	frame=tb,
	language=Java,
	commentstyle=\color{gray},
	keywordstyle=\color{blue},
	stringstyle=\color{red}
}}
% \switchToJava{}{\small}

\newcommand{\switchToProlog}{
\lstset{
	numberstyle=\footnotesize\color{black},
	basicstyle=\ttfamily,
	breakatwhitespace=true,
	breaklines=true,
	captionpos=b,
	keepspaces=true,
	numbers=left,
	numbersep=7pt,
	showspaces=false,
	showstringspaces=false,
	showtabs=false,
	tabsize=2,
	frame=tb,
	language=Java,
	commentstyle=\color{black},
	keywordstyle=\color{black},
	stringstyle=\color{black}
}}
% \switchToProlog{}
\begin{document}
	
\title{Title}
\author{Filippo Nicolini}
\date{121/12/2019}

\newgeometry{margin=0.8in}
\begin{titlepage}
    \begin{center}
        % \vspace*{0.2cm}
        
        \large
        \textbf{ALMA MATER STUDIORUM -- UNIVERSITÀ DI BOLOGNA \\ CAMPUS DI CESENA}
    	\\
        \noindent\hrulefill
        \vspace{0.4cm}
        
        \Large
        Scuola di Ingegneria e Architettura \\
        Corso di Laurea Magistrale in Ingegneria e Scienze Informatiche
        
        \Huge
        \vspace{4cm}
        \textbf{Simulazione di agenti BDI\\
        basati su Prolog in Alchemist}

        \large
        \vspace{1cm}
        Tesi di laurea magistrale
        
        \vspace{5.5cm}
        \begin{minipage}[t]{0.64\textwidth}
            \begin{flushleft}
                \textit{Relatore} 
                \\ 
                \textbf{Chiar.mo Prof.} \textbf{Andrea Omicini} 
                \\
                \vspace{0.4cm}
                \textit{Correlatori} 
                \\ 
                \textbf{Dott. Ing.} \textbf{Danilo Pianini}
                \\
                \textbf{Dott.} \textbf{Giovanni Ciatto}
            \end{flushleft}
        \end{minipage}
        \begin{minipage}[t]{0.34\textwidth}
            \begin{flushright}
                \textit{Candidato} 
                \\ 
                \textbf{Filippo Nicolini}
            \end{flushright}
        \end{minipage}\\
        
        \vfill
        \noindent\hrulefill
        \vspace{0.3cm}
        \Large
        
        Seconda Sessione di Laurea
        \\
        Anno Accademico 2018-2019
    \end{center}
\end{titlepage}
\restoregeometry

\frontmatter

%\pagenumbering{arabic} % serve per mettere i numeri romani
\chapter*{\hl{Abstract}} % crea l'introduzione (un capitolo non numerato)

% imposta l'intestazione di pagina
\rhead[\fancyplain{}{\bfseries ABSTRACT}]{\fancyplain{}{\bfseries\thepage}}
\lhead[\fancyplain{}{\bfseries\thepage}]{\fancyplain{}{\bfseries ABSTRACT}}

% aggiunge la voce Introduzione nell'indice
\addcontentsline{toc}{chapter}{Abstract}

In questo lavoro di tesi si vuole presentare un nuovo linguaggio che fonda le sue basi su AgentSpeak e tuProlog permettendo quindi di implementare il modello ad agenti, ereditato da AgentSpeak, su ambienti o piattaforme diverse, grazie alla flessibilità di tuProlog. Verrà preso in esame l'utilizzo del linguaggio in combinazione con il meta-simulatore Alchemist e ne verranno descritti i dettagli implementativi.

% non numera l'ultima pagina sinistra
\clearpage{\pagestyle{empty}\cleardoublepage}

%crea l'indice
\tableofcontents

% imposta l'intestazione di pagina
\rhead[\fancyplain{}{\bfseries\leftmark}]{\fancyplain{}{\bfseries\thepage}}
\lhead[\fancyplain{}{\bfseries\thepage}]{\fancyplain{}{\bfseries INDICE}}

% non numera l'ultima pagina sinistra
\clearpage{\pagestyle{empty}\cleardoublepage}

% crea l'elenco delle figure
\listoffigures

% non numera l'ultima pagina sinistra
\clearpage{\pagestyle{empty}\cleardoublepage}

% crea l'elenco delle tabelle
%\listoftables

% non numera l'ultima pagina sinistra
\clearpage{\pagestyle{empty}\cleardoublepage}

% crea l'elenco dei codici sorgenti
\lstlistoflistings

% non numera l'ultima pagina sinistra
\clearpage{\pagestyle{empty}\cleardoublepage}

%comando per impostare l'interlinea
\linespread{1.3}\selectfont

\mainmatter

% crea il CAPITOLO
\chapter{Introduzione}
% imposta l'intestazione di pagina
\lhead[\fancyplain{}{\bfseries\thepage}]{\fancyplain{}{\bfseries\rightmark}}
In questo lavoro di tesi si vuole realizzare un nuovo linguaggio ad agenti, che si ispira al modello BDI di AgentSpeak, che permetta di essere utilizzato in ambienti o piattaforme differenti, ovvero sia simulati che reali.

\paragraph{Obiettivo del lavoro.}
L'obiettivo del lavoro di tesi è quello utilizzare la definizione di agenti BDI fatta da AgentSpeak per definire un nuovo linguaggio ad agenti al quale, inoltre, si è voluto aggiungere anche una caratteristica di flessibilità. Quest'ultima è stata raggiunta grazie all'utilizzo della libreria tuProlog che ha permesso di definire un linguaggio che possa essere utilizzato da interpreti realizzati su ambienti e piattaforme differenti accomunate dall'utilizzo di questa libreria.

AgentSpeak è un linguaggio orientato agli agenti basato sulla programmazione logica e l'architettura BDI per programmare agenti autonomi. Il modello BDI (Beliefs, Desires, Intentions) implementa gli aspetti principali del ragionamento umano e consente di programmare agenti intelligenti.
tuProlog è una libreria che permette di sfruttare il linguaggio Prolog, incapsulato in un core minimale, all'interno di applicazioni e infrastrutture distribuite.

Si vuole mostrare, inoltre, come è possibile creare un interprete all'interno del meta-simulatore Alchemist, realizzando un'opportuna incarnazione, che sfrutta il modello di agenti BDI definito da AgentSpeak e l'implementazione di Jason, in particolare relativamente al ciclo di ragionamento, e che permette l'esecuzione di agenti, definiti tramite le teorie utilizzando il nuovo linguaggio, in un ambiente simulato.

Alchemist è un simulatore per il calcolo pervasivo, aggregato e naturale, che si basa su un meta-modello flessibile, il quale permette di realizzare implementazioni di modelli completamente diversi tra loro.
Jason è un interprete di una versione estesa di AgentSpeak che implementa un linguaggio e fornisce una piattaforma per sviluppare sistemi multi-agente.

\paragraph{Benefici dell'approccio scelto.}
Il beneficio del linguaggio risiede intrinsecamente nell'architettura del modello ad agenti BDI poichè cerca di esprimere tutto il potenziale del paradigma ad agenti.
In particolare con la definizione di questo nuovo linguaggio si vuole fornire una soluzione per poter sfruttare a pieno l'implementazione dell'agente separando le sue competenze da quelle, invece, puramente demandate all'interprete.
In questo modo all'interno dell'agente saranno definiti solo i meccanismi di reazione a determinati eventi mentre la parte di scheduling e gestione del modello saranno determinati dall'implementazione dell'interprete.
L'agente quindi si comporterà in modo diverso rispetto a come verrà realizzato l'interprete, ovvero in base all'implementazione di:
\begin{itemize}
\item funzioni di prelazione relative a selezione di piani e intenzioni
\item gestione dei messaggi
\item gestione dell'ambiente esterno
\item selezione degli eventi da far gestire all'agente
\end{itemize}

\paragraph{Struttura dell'elaborato.}
In questo capitolo è stato descritto l'obiettivo del lavoro ed i benefici che si vogliono ottenere.
A seguire è accennato il contenuto dei prossimi capitoli presenti in questo lavoro di tesi.

Nel \cref{chap:soa} sono descritti inizialmente i lavori correlati al modello ad agenti, ai sistemi multi-agente e sistemi distribuiti, a cui segue una descrizione del modello ad agenti formalizzato da AgentSpeak e del ciclo di ragionamento utilizzato dall'interprete Jason.
Successivamente sono descritte le caratteristiche di tuProlog, Alchemist e del modello SpatialTuples, il quale definisce un'estensione del modello base di tuple per i sistemi multi-agente.

Nel \cref{chap:agentspeak-2p} è definito il nuovo linguaggio ad agenti, il quale unisce la solidità del modello AgentSpeak alle semantiche operazionali del ciclo di ragionamento Jason, e le relative API messe a disposizione per poter programmare l'operato di un agente.
All'interno del capitolo è descritto come avviene la gestione delle intenzioni nella teoria dell'agente ed inoltre sono presenti alcuni esempi base che mostrano il funzionamento del linguaggio.

Nel \cref{chap:agentspeak-2p-alchemist} è iniziata la descrizione dell'implementazione dell'interprete che, per questo progetto di tesi, è stato realizzato all'interno del simulatore Alchemist.
Inizia dall'analisi del mapping per proseguire con la descrizione della libreria, implementata per gestire le primitive del linguaggio che è stato definito, e di alcune situazioni, come la gestione delle azioni e lo spostamento del nodo, in cui questa è stata utilizzata per la gestire il collegamento tra l'interprete e la teoria.

Nel \cref{chap:impl} si è scesi più nello specifico dell'implementazione dell'interprete descrivendo come sono state create le varie classi, in modo particolare le entità Incarnation, Node e Action.
Successivamente è mostrato un ulteriore dettaglio sull'implementazione di alcune funzionalità peculiari dell'interprete, quali la gestione dei messaggi attraverso l'agente Postman, la referenziazione di oggetti Java nelle teorie tramite tuProlog e il modo in cui sono create le intenzioni.
A seguire sono accennate brevemente le caratteristiche dei tool di sviluppo utilizzati e di come scrivere la configurazione di una simulazione.

Nel \cref{chap:validation} è descritto uno scenario di test noto nei sistemi multi-agente, il problema dei Goldminers, e sono descritte le classi e le teorie implementate per poterlo realizzare.
All'interno del capitolo è presente una sezione per spiegare come è stato possibile tematizzare la simulazione; grazie allo stile creato è stato possibile controllare lo sviluppo dello scenario nel tempo in maniera più semplice.
Nella sezione delle metriche di progetto, sono descritti alcuni indicatori che permettono di analizzare meglio il lavoro svolto.

Infine, nel \cref{chap:conclusions}, sono descritti e analizzati i risultati ottenuti sia attraverso il linguaggio che tramite la definizione dell'interprete.

% crea il CAPITOLO
\chapter{Alchemist}
% imposta l'intestazione di pagina
\lhead[\fancyplain{}{\bfseries\thepage}]{\fancyplain{}{\bfseries\rightmark}}
% mette i numeri arabi
%\pagenumbering{arabic}

Per la realizzazione dell'ambiente per il progetto di tesi è stato scelto il simulatore Alchemist, poichè fornisce già una struttura ed un meta-modello solido e utilizzabile. Nel capitolo viene mostrato Alchemist attraverso le funzionalità e ciò che lo compone. Successivamente è presentata una serie di peculiarità di questo strumento che saranno approfondite più avanti nel documento.

%----------------------------
\section{Descrizione Alchemist}
Alchemist è un simulatore per il calcolo pervasivo, aggregato e ispirato alla natura. Esso fornisce un ambiente di simulazione sul quale è possibile sviluppare nuove incarnazioni, cioè nuove definizioni di modelli implementati su di esso. Ad oggi sono disponibili le funzionalità per:
\begin{itemize}
\item simulare un ambiente bidimensionale;
\item simulare mappe del mondo reale, con supporto alla navigazione e importazione di tracciati in formato gpx;
\item simulare ambienti indoor importando immagini in bianco e nero;
\item eseguire simulazioni biologiche utilizzando reazioni in stile chimico;
\item eseguire programmi Protelis, Scafi, SAPERE (scritti in un linguaggio basato su tuple come LINDA).
\end{itemize}

\subsection{Meta-modello}
Il meta-modello di Alchemist può essere compreso osservando la figura \ref{fig:alchemistModel}.
%crea l'ambiente figura;
\begin{figure}[h] % [h] sta per here, cioè la figura va qui
\begin{center} % centra nel mezzo della pagina la figura
\includegraphics[width=12.5cm]{images/AlchemistModel.png} % inserisce una figura larga 12.5cm
% inserisce la legenda ed etichetta la figura con \label{fig:prima}
\caption[Illustrazione meta-modello di Alchemist]{Illustrazione meta-modello di Alchemist} \label{fig:alchemistModel}
\end{center}
\end{figure}

L'\textbf{\textit{Environment}} è l'astrazione dello spazio ed è anche l'entità più esterna che funge da contenitore per i nodi. Conosce la posizione di ogni nodo nello spazio ed è quindi in grado di fornire la distanza tra due di essi e ne permette inoltre lo spostamento.

\`E detta \textbf{\textit{Linking rule}} una funzione dello stato corrente dell'environemnt che associa ad ogni nodo un \textbf{\textit{Vicinato}}, il quale è un entità composta da un nodo centrale e da un set di nodi vicini.

Un \textbf{\textit{Nodo}} è un contenitore di molecole e reazioni che è posizionato all'interno di un environment.

La \textbf{\textit{Molecola}} è il nome di un dato, paragonabile a quello che rappresenta il nome di una variabile per i linguaggi imperativi.
Il valore da associare ad una molecola è detto \textbf{\textit{Concentrazione}}.

%crea l'ambiente figura;
\begin{figure}[h] % [h] sta per here, cioè la figura va qui
\begin{center} % centra nel mezzo della pagina la figura
\includegraphics[width=14cm]{images/AlchemistReaction.png} % inserisce una figura larga 12.5cm
% inserisce la legenda ed etichetta la figura con \label{fig:prima}
\caption[Illustrazione modello reazione di Alchemist]{Illustrazione modello reazione di Alchemist} \label{fig:alchemistReaction}
\end{center}
\end{figure}

Una \textbf{\textit{Reazione}} è un qualsiasi evento che può cambiare lo stato dell'environment ed è definita tramite una distribuzione temporale, una lista di condizioni e una o più azioni.
\\La frequenza con cui avvengono dipende da:
\begin{itemize}
\item un parametro statico di frequenza;
\item il valore di ogni condizione;
\item un'equazione di frequenza che combina il parametro statico e il valore delle condizioni restituendo la frequenza istantanea;
\item una distribuzione temporale.
\end{itemize}
Ogni nodo contiene un set di reazioni che può essere anche vuoto.

Per comprendere meglio il meccanismo di una reazione si può osservare la figura \ref{fig:alchemistReaction}.

Una \textbf{\textit{Condizione}} è una funzione che prende come input l'environment corrente e restituisce come output un booleano e un numero. Se la condizione non si verifica, le azioni associate a quella reazione non saranno eseguite. In relazione a parametri di configurazione e alla distribuzione temporale, una condizione potrebbe influire sulla velocità della reazione.

La \textbf{\textit{Distribuzione temporale}} indica il numero di eventi, in un dato intervallo di tempo, generati da Alchemist e che innescano la verifica delle condizioni che possono portare alla potenziale esecuzione delle azioni.

Un'\textbf{\textit{Azione}} è la definizione di una serie di operazioni che modellano un cambiamento nel nodo o nell'environment.

In Alchemist un'incarnazione è un'istanza concreta del meta-modello appena descritta e che implementa una serie di componenti base come: la definizione di una molecola e del tipo di dati della concentrazione, un set di condizioni, le azioni e le reazioni. Incarnazioni diverse possono modellare universi completamente differenti.


\section{Aspetti principali in Alchemist}
Alchemist è uno strumento molto esteso e che può offrire tantissime possibilità se lo si conosce e si è in grado di padroneggiarlo.
La conoscenza iniziale di questo ambiente era però praticamente nulla e, quindi è stato necessario impiegare del tempo per riuscire a padroneggiare i meccanismi di base, tra cui l'utilizzo delle classi delle entità del meta-modello e la scrittura della configurazione di una simulazione.

Gli aspetti principali di Alchemist sono la forte adattabilità del meta-modello sul quale è costrutito, il numero di implementazioni o astrazioni base già disponibili e la grande personalizzazione delle simulazioni.
\\
Il meta-modello fornisce una struttura molto solida sulla quale è possibile realizzare ambiti applicativi anche molto diversi tra loro.
\\
Inoltre, Alchemist, come citato poco fa, fornisce già una serie di implementazioni o astrazioni di classi relative a entità del meta-modello che permettono di avviare lo sviluppo di un'incarnazione in modo molto più rapido.
\\
Per quanto riguarda le simulazioni, queste sono realizzate tramite una configurazione che consiste in una mappa definita tramite il linguaggio YAML. La mappa è composta da diverse sezioni, ognuna caratterizzata da una specifica keyword, ed è altamenta configurabile: questo permette all'utente di testare tantissimi aspetti della simulazione.







\chapter{AgentSpeak in tuProlog}
Nel capitolo precedente è stato descritto lo stato attuale di lavori correlati che utilizzano il modello ad agenti BDI per costruirne altri più complessi ed espressivi o implementano linguaggi basati sugli agenti. Inoltre, è stato mostrato lo stato dell'arte delle tecnologie che sono state utilizzate.

In questo lavoro di tesi si è voluto definire un nuovo linguaggio che fosse \textit{`platform indipendent'}, ovvero indipendente dall'ambiente sul quale viene utilizzato: è stato definito al pari di una libreria, senza nessun riferimento all'ambiente. Nei capitoli successivi verrà mostrato come, partendo dal linguaggio, è stata colmata la distanza con l'ambiente scelto.

\section{Definizione linguaggio}\label{sctn:definizioneLinguaggio}
Essendo tuProlog un interprete che opera su piattaforme differenti, si è voluto utilizzarlo nella definizione del linguaggio per permettere di utilizzare quest'ultimo facilmente, potendo sfruttare la libreria tuProlog per colmare la distanza tra l'ambiente e il linguaggio.

Seguendo quella che è la struttura di AgentSpeak, è stato formalizzato questo linguaggio di programmazione ad agenti, e, di seguito, sono mostrate le definizioni.

\smallskip
% Definizione 1
\begin{defn}
Se $b$ è un simbolo di predicato e $t_1, \ldots, t_n$ sono termini, allora $belief(b(t_1, \ldots, t_n))$ è un atomo di belief.
Se $belief(b(t))$ e $belief(c(s))$ sono atomi di belief, allora $belief(b(t)) \land belief(c(s))$ e $\neg belief(b(t))$ sono beliefs.
Un atomo di belief oppure la sua negazione sono riferiti al letterale del belief. Un atomo di belief base sarà chiamato \textit{belief base}.
\end{defn}

\smallskip
% Definizione 2
\begin{defn}\label{defn:goals}
Se $g$ è un simbolo di predicato e $t_1, \ldots, t_n$ sono termini, allora $achievement(g(t_1, \ldots, t_n))$ e $test(g(t_1, \ldots, t_n))$ sono \textit{goals}.
\end{defn}

\smallskip
% Definizione 3
\begin{defn}\label{defn:triggeringEvents}
Se $b(t)$ è un atomo di belief e $g(t)$ un goal, allora $onAddBelief(b(t))$, $onRemoveBelief(b(t))$, $onReceivedMessage(b(t))$, $onResponseMessage(b(t))$, $concurrent(achievement(g(t)))$, $concurrent(test(g(t)))$ sono \textit{eventi di attivazione}.
\end{defn}

\smallskip
% Definizione 4
\begin{defn}
Se $a$ è un simbolo di azione e $t_1, \ldots, t_n$ sono termini del primo ordine, allora $a(t_1, \ldots, t_n)$ è un'azione.
\end{defn}

\smallskip
% Definizione 5
\begin{defn}
%Se $e$ è un \textit{evento di attivazione} e $h_1, \ldots, h_n$ sono goals o azioni, allora $e :- $h_1, \ldots, h_n$ è un piano. L'espressione a sinistra di `:-' è la testa, quella sulla destra il corpo: se quest'ultimo è vuoto viene definito con l'espressione \textit{true}.
Se $e$ è un \textit{evento di attivazione}, $b_1, \ldots, b_m$ sono belief o guardie e $h_1, \ldots, h_n$ sono goals o azioni, allora $`\leftarrow'(e, [b_1, \ldots, b_m], [h_1, \ldots, h_n])$ è un piano.
\end{defn}

\smallskip
% Definizione 6
\begin{defn}\label{defn:intenzione}
Ogni intenzione ha al suo interno uno stack di piani parzialmente istanziati, ovvero dove alcune delle variabili sono state istanziate. Un'intenzione è definita come $intention(i, [p_1, \ldots, p_n])$, dove $i$ è l'identificativo univoco dell'intenzione e $[p_1, \ldots,p_n]$ è lo stack formata da azioni, belief o goal: $p_1$ è la coda e $p_n$ è la testa.
\end{defn}

\smallskip
% Definizione 7
\begin{defn}
Un \textit{agente} è formato da $\langle B,P,I,A,S_O,S_I \rangle$, dove $B$ è una `belief base', $P$ è un set di piani, $I$ è un set di intenzioni, $A$ è un set di azioni. La funzione $S_O$ sceglie un piano dal set di quelli applicabili; la funzione $S_I$ sceglie l'intenzione da eseguire dal set $I$.
\end{defn}

\smallskip
% Definizione 8
\begin{defn}
Dato un evento $\epsilon$ ed un piano $p = `\leftarrow'(e, [b_1, \ldots, b_m], [h_1, \ldots, h_n])$, allora $p$ è rilevante per l'evento $\epsilon$ se e solo se esiste un unificatore $\sigma$ tale per cui $d\sigma = e\sigma$. $\sigma$ è detto \textit{unificatore rilevante} per $\epsilon$.
\end{defn}

\smallskip
% Definizione 9
\begin{defn}
Un piano $p$ è definito da $`\leftarrow'(e, [b_1, \ldots, b_m], [h_1, \ldots, h_n])$ è un \textit{piano applicabile} rispetto ad un evento $\epsilon$ se e solo se esite un identificatore rilevante $\sigma$ per $\epsilon$ e esiste una sostituzione $\theta$ tale che $\forall (b_1, \ldots, b_m) \sigma\theta$ è una conseguenza logica di $B$.
%La composizione $\sigma\theta$ è riferita all'\textit{unificatore applicabile} per l'evento $\epsilon$ e $\theta$ è riferita alla sostituzione della corretta risposta.
\end{defn}

%\bigskip
%TODO
%
%\textbf{selezione intenzione (13,14,15,16)}
%
%\smallskip
%% Definizione 13
%\begin{defn}
%Sia $S_I(I) = i$, dove $i$ è $[p_1 \ddagger \ldots \ddagger f : c_1 \land \ldots \land c_y \leftarrow !g(t);h_2; \ldots; h_n]$. L'intenzione $i$ si dice che è eseguita  se e solo se $\langle +!g(t), i \rangle \in E$.
%\end{defn}
%
%\smallskip
%% Definizione 14
%\begin{defn}
%Sia $S_I(I) = i$, dove $i$ è $[p_1 \ddagger \ldots \ddagger f : c_1 \land \ldots \land c_y \leftarrow ?g(t); h_2; \ldots; h_n]$. L'intenzione $i$ si dice che è eseguita  se e solo se esiste una sostituzione $\theta$ tale che $\forall g(t) \theta$ è una conseguenza logica di B e $i$ è rimpiazzato da $[p_1 \ddagger \ldots \ddagger (f : c_1 \land \ldots \land c_y) \theta \leftarrow h_2 \theta; \ldots; h_n \theta]$.
%\end{defn}
%
%\smallskip
%% Definizione 15
%\begin{defn}
%Sia $S_I(I) = i$, dove $i$ è $[p_1 \ddagger \ldots \ddagger f : c_1 \land \ldots \land c_y \leftarrow a(t); h_2; \ldots; h_n]$. L'intenzione $i$ si dice che è eseguita  se e solo se $a(t) \in A$, e $i$ è rimpiazzato da $[p_1 \ddagger \ldots \ddagger f : c_1 \land \ldots \land c_y \leftarrow h_2; \ldots; h_n]$.
%\end{defn}
%
%\smallskip
%% Definizione 16
%\begin{defn}
%Sia $S_I(I) = i$, dove $i$ è $[p_1 \ddagger \ldots \ddagger p_{z-1} \ddagger g(t) : c_1 \land \ldots \land c_y \leftarrow true]$, dove $p_{z-1}$ è $e : b_1 \land \ldots \land b_x \leftarrow !g(s); h_2; \ldots; h_n$. L'intenzione $i$ si dice che è eseguita  se e solo se esiste una sostituzione $\theta$ tale che $g(t)\theta = g(s)\theta$ e $i$ è rimpiazzato da $[p_1 \ddagger \ldots \ddagger p_{z-1} \ddagger (e : b_1 \land \ldots \land b_x)\theta \leftarrow (h_2; \ldots; h_n) \theta]$.
%\end{defn}

\section{API del linguaggio}
Il linguaggio appena definito è ciò che è messo a disposizione del programmatore dell'agente per descrivere il suo comportamento. Oltre questo, sono state definite altre sintassi che permettano al programmatore di gestire ogni evento o situazione per l'agente.
Qui di seguito sono citate regole, variabili e fatti del linguaggio:
\begin{itemize}
\item $init :- \ldots$
\item $self(A).$
\item $agent$
\item $node$
\item $addBelief(B).$
\item $removeBelief(B).$
\item $onAddBelief(B) :- \ldots$
\item $onRemoveBelief(B) :- \ldots$
\item $onReceivedMessage(S, M) :- \ldots$
\item $achievement(t).$
\item $test(t).$
\item $concurrent(t).$
\item $belief(position(X,Y)).$
\item $belief(distance(A, ND, OD)).$ oppure $belief(distance(A, ND)).$
\end{itemize}

Di seguito vengono analizzate ed esposte.
\\
Per ogni regola è lasciata l'implementazione del corpo al programmatore dell'agente.

La regola $`init'$ è messa a disposizione per permettere di far effettuare una configurazione iniziale dell'agente. Infatti, questa regola verrà invocata solo ed esclusivamente la prima volta che viene attivato l'agente, al posto del ciclo di ragionamento. In questo modo il programmatore dell'agente è in grado di far eseguire all'agente una serie di operazioni iniziali per impostare ad esempio la `belief base' dell'agente.

\medskip
Il fatto $`self(A)'$ permette all'agente di recuperare il suo nome. In questo modo il nome dell'agente può essere recuperato anche all'interno della teoria tuProlog.

\medskip
I due letterali $`agent'$ e $`node'$ sono due variabili di tuProlog alle quali sono collegati gli oggetti dell'agente e del nodo, implementati nell'ambiente su cui si è scelto di utilizzare il linguaggio. Se costruiti correttamente, dalla teoria dell'agente sarà possibile richiamare metodi implementati nella classe corrispondente. La variabile $`agent'$ fa riferimento all'oggetto dell'agente stesso, mentre $`node'$ si riferisce all'oggetto che rappresenta lo spazio sul quale l'agente è inserito. In questo modo possono essere gestite le azioni interne ed esterne dell'agente.

\medskip
Le regole $`addBelief(B)'$ e $`removeBelief(B)'$ sono utilizzabili per aggiungere o rimuovere elementi dalla `belief base'. Il loro utilizzo scatena un evento che va ad invocare $`onAddBelief(B)'$, $`onRemoveBelief(B)'$. Più precisamente $`onAddBelief(B)'$ viene invocato quando viene aggiunto un belief, mentre $`onRemoveBelief(B)'$ è chiamato in seguito alla rimozione di un belief dalla `belief base'. In entrambi i casi la variabile $B$ corrisponde al belief inserito o rimosso.
\\
Diversamente, quando viene letto un messaggio ricevuto da un altro agente (o anche da se stesso), è invocato $`onReceivedMessage(S, M)'$, dove $S$ rappresenta il mittente e $M$ il contenuto del messaggio, che consente all'agente di reagire quando viene letto un messaggio tra quelli presenti nella sua coda di ingresso.

\medskip
Come visto precedentemente nella Definizione \ref{defn:goals}, i letterali $`achievement'$, $`test'$ sono utilizzati per impostare dei goal nell'agente. Ciò che viene scatenato  è l'inserimento della serie di operazioni definita dal goal in testa allo stack dell'intenzione.
In combinazione, i due letterali appena citati possono essere usati in combinazione con $`concurrent'$, mostrato nella Definizione \ref{defn:triggeringEvents}, che permette di inserire le operazioni definite nel goal in una nuova intenzione. In questo modo, la nuova intenzione può essere eseguita in modo concorrente o parallelo rispetto a quella `padre'.

\medskip
Per rendere disponibile al programmatore dell'agente varie possibilità per accedere a informazioni quali la posizione dell'agente e la distanza degli altri agenti rispetto alla propria posizione, sono utilizzati due belief che saranno aggiornati direttamente dall'ambiente sul quale viene utilizzato il linguaggio. La posizione dell'agente viene aggiornata una volta per ogni ciclo di ragionamento e, al termine, sono modificati anche i valori dei belief appena citati. 
\\
Per quanto riguarda la posizione dell'agente, potrà essere invocato $`belief(position(X, Y))'$ dove $X$ è la coordinata relative alle ascisse o longitudine e $Y$ è la coordinata relativa alle ordinate o latitudine.
\\
La distanza da altri agenti può essere molto utile per far scegliere all'agente di effettuare o meno una certa azione. Se nella lista del vicinato entra un nuovo agente viene inserito il belief $`belief(distance(A, ND))'$, dove $A$ è il nome dell'agente nel vicinato e $ND$ è la distanza che li separa. Se, invece, un agente era già nella lista del vicinato e vi rimane, allora viene inserito il belief $`belief(distance(A, ND, OD))'$, dove $A$ è il nome dell'agente nel vicinato, $ND$ è la nuova distanza che li separa e $OD$ è la distanza che li divideva precedentemente.

\subsection{Gestione intenzioni}
Le intenzioni sono la modalità con cui l'agente opera le sue azioni. Come descritto in precedenza, nel ciclo di ragionamento alla sezione \ref{ssctn:cicloRagionamentoAgentSpeak}, l'agente esegue una serie di passi che portano all'esecuzione di un'azione.
Qui di seguito è descritto come avviene il ciclo di ragionamento utilizzando questo linguaggio. La spiegazione terrà conto solamente degli aspetti relativi alla parte tuProlog e quindi sarà incompleta fino al raggiungimento della sezione \ref{sctn:interpreteLinguaggio}. Le funzioni di selezione per i piani applicabili e le intenzioni non sono trattate in questa parte, poichè sono relative all'implementazione dell'interprete.

L'agente lato tuProlog definisce il suo comportamento tramite una serie di regole e fatti che gli permettono di reagire ad eventi sia esterni che interni. Una percezione dell'ambiente può essere scatenata ad esempio da uno spostamento o una modifica della `belief base': quando questo avviene l'ambiente sul quale è utilizzato il linguaggio invoca una delle regole che, se implementata correttamente nella teoria dell'agente, consente all'agente di reagire all'evento. Un altro tipo di input che può ricevere l'agente è la ricezione di un messaggio. In tuProlog l'agente può reagire alla lettura del contenuto del messaggio poichè l'implementazione e la gestione delle code e la selezione dei messaggi viene fatta dall'interprete.

La frequenza dell'esecuzione del ciclo di ragionamento dipende dall'ambiente sul quale viene utilizzato il linguaggio. All'interno del ciclo, una volta selezionato il piano applicabile per l'evento avvenuto, l'interprete invoca delle regole per ottenere la lista delle operazioni presenti nel corpo del piano (o regola) per poterle inserire nell'intenzione. Le regole per recuperare la lista eseguono per ogni elemento del corpo una lettura e un inserimento all'interno di una lista, la quale poi viene restituita.

La creazione dell'intenzione viene fatta dall'interprete ma salvata come fatto nella teoria dell'agente. Come descritto nella definizione \ref{defn:intenzione}, l'intenzione $intention(id,[op_1, \ldots, op_n])$ è composta da un identificativo univoco $id$ e da una lista di operazioni $[op_1, \ldots, op_n]$. All'interno della teoria dell'agente possono essere presenti più intenzioni contemporaneamente ma ad ogni ciclo di ragionamento solo una verrà selezionata per l'esecuzione. Come detto precedentemente, anche la selezione dell'intenzione è gestita dall'interprete ma, lato tuProlog viene gestita l'esecuzione. Infatti, l'interprete si limita a invocare la regola $execute(I)$ all'interno della quale viene gestita l'esecuzione della prima operazione sullo stack dell'intenzione con identificativo $I$, ovvero quella che è stata precedentemente selezionata.
\\
La regola $execute$ si occupa di recuperare l'intenzione riferita all'identificativo passato e quindi prendere la testa dello stack delle operazioni. Quest'ultima viene valutata ed in base alla sua natura vengono eseguite azioni diverse:
\begin{itemize}
\item le azioni vengono eseguite direttamente;
\item i goal vengono recuperati lo stack di operazioni collegate viene successivamente aggiunto in testa all'intenzione di cui faceva parte il goal;
\item i goal espressi all'interno di $concurrent$ creano una nuova intenzione che potrà essere eseguita in parallelo rispetto a quella da cui ha avuto origine la chiamata al goal.
\end{itemize}


\subsection{Estensione Spatial Tuples}
Il linguaggio appena descritto è stato esteso per permettere di utilizzare il modello Spatial Tuples. Sono state quindi inserite le seguenti regole:
\begin{itemize}
\item $writeTuple(T).$
\item $readTuple(TT).$
\item $takeTuple(TT).$
\item $onResponseMessage(M) :- \ldots$
\end{itemize}
Le regole dell'elenco sono tutte riferite all'inserimento, all'interno del linguaggio, del modello di coordinazione LINDA e più precisamente del modello Spatial Tuples.
Con questa estensione, viene data la possibilità agli agenti di poter inserire e recuperare informazioni posizionate nello spazio. Le regole messe a disposizione mappano le primitive dei modelli che vogliono implementare `\textit{in}', `\textit{rd}', `\textit{out}' rispettivamente in $`writeTuple(T)'$, $`readTuple(TT)'$, $`takeTuple(TT)'$ dove $T$ è la tupla da inserire e $TT$ è il template da ricercare.
\\
Utilizzando $`writeTuple(T)'$ il programmatore è in grado di inserire informazioni posizionate nello spazio degli agenti e con le quali gli stessi agenti possono interagire. Per leggere le informazioni si possono utilizzare due diverse modalità: $`readTuple(TT)'$ e $`takeTuple(TT)'$. Nel primo caso viene utilizzato il template passato per confrontarlo con le tuple nell'intorno dell'agente e se ci sono risultati che combaciano con il template allora uno di questi viene restituito. Per quanto riguarda invece $`takeTuple(TT)'$, si comporta ugualmente per quanto riguarda la ricerca della tupla con il template ma poi, una volta trovati i risultati ne sceglie uno e prima di restituirlo lo elimina dallo spazio di tuple in cui era presente.
\\
Entrambe le due modalità di getter seguono la semantica standard dei modelli basati su tuple, e quindi sono:
\begin{itemize}
\item sospensive: se non ci sono tuple che si abbinano al template l'operazione è bloccata finchè non viene trovata una tupla;
\item non deterministiche: se ci sono più tuple che si abbinano al template una è scelta in modo non deterministico.
\end{itemize}
Per dare la possibilità di gestire la risposta e gestire la tupla restituita è stata introdotta $`onResponseMessage(M)'$ che viene invocata ogni qualvolta che una tupla viene restituita dallo spazio di tuple. Il contenuto $M$ è la tupla incapsulata in un belief in modo che si possano gestire tuple provenienti da diversi spazi di tuple e con contenuti differenti.

\section{Esempi linguaggio}
In questa sezione verranno mostrate alcuni casi d'uso del linguaggio appena descritto. Nello specifico verrà mostrato un primo scenario dove sono stati configurati gli agenti per realizzare un semplice scambio di messaggi (o Ping Pong). Nel secondo esempio, invece, viene illustrato come poter utilizzare l'estensione Spatial Tuples supportata dal linguaggio.

\subsubsection{Ping Pong}
In questo primo esempio è presentato il problema del Ping Pong. In questo esempio sono definiti due agenti, Ping e Pong, ognuno dei quali risponde ad un messaggio ricevuto. L'agente Ping, alla ricezione del messaggio `\textit{pong}' da parte dell'agente Pong risponderà con un messaggio `\textit{ping}'. Viceversa, l'agente Pong, alla ricezione del messaggio `\textit{ping}' da parte dell'agente Ping risponderà con un messaggio `\textit{pong}'.

Per far iniziare lo scambio di messaggi è stato utilizzato `init' per impostare all'interno di uno dei due agenti, nello specifico l'agente Ping, un'intenzione iniziale. In questo modo, al primo ciclo di ragionamento, l'agente eseguirà l'intenzione e invierà il primo messaggio.
\lstset{
  %numberstyle=\footnotesize\color{black},
  basicstyle=\ttfamily,
  %breakatwhitespace=false,
  %breaklines=true,
  captionpos=b,
  %keepspaces=true,
  %numbers=left,
  %numbersep=0pt,
  %showspaces=false,
  %showstringspaces=false,
  %showtabs=false,
  frame=tb,
  %commentstyle=\color{black},
  %keywordstyle=\color{black},
  %stringstyle=\color{black}
  %label=incarnationYAML,
  %caption={First verbatim}
  %language=Java
  %escapeinside={(*@}{@*)}
}
\medskip
\begin{lstlisting}[firstnumber=1,label={lst:PingAgent},caption={Agente Ping}]
init :-
  addBelief(intention(0,[iSend('pong_agent','ping')])),
  agent <- insertIntention(0).

onReceivedMessage(S,pong) :-
  iSend(S, ping).
\end{lstlisting}

In entrambe le teorie dei due agenti è stata richiamata $`iSend(S, M)'$, dove $S$ è il destinatario e $M$ è il messaggio, che è un'azione interna dichiarata e gestita nell'ambiente sul quale è utilizzato il linguaggio. Nel Codice sorgente \ref{lst:PingAgent} viene inviato all'agente Pong il messaggio `ping', mentre nel Codice sorgente \ref{lst:PongAgent} il messaggio inviato all'agente Ping è `pong'.

\medskip
\begin{lstlisting}[firstnumber=1,label={lst:PongAgent},caption={Agente Pong}]
onReceivedMessage(S,ping) :-
  iSend(S, pong).
\end{lstlisting}

\subsubsection{Message passing through Spatial Tuples}
In questo esempio viene mostrato come possono essere utilizzate le primitive del modello Spatial Tuples incorporate nel linguaggio descritto in precedenza. Nello specifico viene mostrato come tre agenti (Alice, Bob e Carl) comunicano tra loro inserendo messaggi negli spazi di tuple a loro vicini, usandoli come `lavagna'.
L'agente Alice nel suo ciclo di configurazione, esegue due scritture sulla `lavagna' (spazio di tuple) inserendo messaggi per Bob e Carl e successivamente effettua altre due richieste allo spazio di tuple richiedendo due messaggi a lei destinati senza conoscerne il contenuto. Una volta ricevuti i messaggi non fa niente.

\medskip
\begin{lstlisting}[firstnumber=1,label={lst:Alice},caption={Alice}]
init :-
  writeTuple(blackboard,msg(bob,hello)),
  writeTuple(blackboard,msg(carl,hello)),
  takeTuple(blackboard,msg(alice,X)),
  takeTuple(blackboard,msg(alice,X)).

onResponseMessage(msg(X,Y)) :- true.
\end{lstlisting}

L'agente Bob, nel suo ciclo di configurazione, effettua una richiesta allo spazio di tuple per ricevere messaggi a lui destinati. Inoltre, nella sua teoria, è definito un comportamento in caso di ricezione del messaggio: manda ad Alice lo stesso messaggio che ha ricevuto.
\medskip
\begin{lstlisting}[firstnumber=1,label={lst:Bob},caption={Bob}]
init :-
  takeTuple(blackboard,msg(bob,X)).

onResponseMessage(msg(bob,X)) :-
  writeTuple(blackboard,msg(alice,X)).
\end{lstlisting}

Come Bob, l'agente Carl esegue lo stesso comportamento di Bob.
\medskip
\begin{lstlisting}[firstnumber=1,label={lst:Carl},caption={Carl}]
init :-
  takeTuple(blackboard,msg(carl,X)).

onResponseMessage(msg(carl,X)) :-
  writeTuple(blackboard,msg(alice,X)).
\end{lstlisting}


% crea il CAPITOLO
\chapter{Agent speak}
% imposta l'intestazione di pagina
\lhead[\fancyplain{}{\bfseries\thepage}]{\fancyplain{}{\bfseries\rightmark}}
% mette i numeri arabi
%\pagenumbering{arabic}

AgentSpeak è un linguaggio orientato agli agenti basato sulla programmazione logica e sul modello BDI (Belief-Desires-Intention). All'interno del capitolo viene descritto cos'è il modello BDI e come funziona AgentSpeak, ed in particolare il ciclo di ragionamento, esemplificato da quello di Jason.

%----------------------------
\section{Agenti BDI con AgentSpeak}
Il modello BDI consente di rappresentare le caratteristiche e le modalità di raggiungimento di un obiettivo secondo il paradigma ad agenti. Gli agenti BDI forniscono un meccanismo per separare le attività di selezione di un piano, fra quelli presenti nella sua teoria, dall'esecuzione del piano attivo, permettendo di bilanciare il tempo speso nella scelta del piano e quello per eseguirlo.

I \textbf{beliefs} sono quindi informazioni dello stato dell'agente, ovvero ciò che l'agente sa del mondo \cite{AgentSpeakInJason} il suo insieme è chiamato `belief base' o `belief set'.

I \textbf{desires} rappresentano tutti i possibili piani che un agente potrebbe eseguire \cite{AgentSpeakInJason}. Rappresentano ciò che l'agente vorrebbe realizzare o portare a termine: i \textit{goals} sono desideri che l'agente persegue attivamente ed è quindi bene che tra loro siano coerenti, cosa che non è obbligatoria per quanto riguarda il resto dei desideri.

Le \textbf{intentions} identificano i piani a cui l'agente ha deciso di lavorare o a cui sta già lavorando e a loro volta possono contenere altri piani \cite{AgentSpeakInJason}.

Gli \textbf{eventi} innescano le attività reattive, ovvero la caratteristica di proattività degli agenti, come ad esempio l'aggiornamento dei beliefs, l'invocazione di piani o la modifica dei goals.

\section{Definizione AgentSpeak}
\textbf{AgentSpeak} è un linguaggio di programmazione basato su un linguaggio del primo ordine con eventi e azioni \cite{AgentSpeak(L)}. Il comportamento degli agenti è dettato da quanto definito nel programma scritto in AgentSpeak. I beliefs correnti di un agente sono relativi al suo stato attuale, all'enviroment e agli altri agenti. Gli stati che un agente vuole determinare sulla base dei suoi stimoli esterni e interni sono i desideri \cite{AgentSpeak(L)}. L'adozione di programmi per soddisfare tali stimoli è detta intenzioni.

\subsection{Composizione}
Un'agente in AgentSpeak è formato da una `belief base' e da una serie di piani opportunamente programmati.
La `belief base' è il contenitore dello stato dell'agente, dove sono presenti tutte le informazioni che esso ha in riferimento a se stesso e all'ambiente. Questo set di belief può essere modificato in modo continuo dalle azioni scatenate nel ciclo di ragionamento.
I piani sono sequenze di azioni o goal che permettono all'agente di reagire a situazioni che avvengono nell'ambiente o internamente.
\\
Qui di seguito viene descritto quali sono le fasi del ciclo di ragionamento dell'agente e in che modo viene definito il suo operato. La sequenza di esecuzione di questa iterazione è fondamentale per la realizzazione di un agente e quindi importante comprenderla per realizzare un'architettura appropriata.

\subsection{Ciclo di ragionamento}\label{ssctn:cicloRagionamentoAgentSpeak}
Il ciclo di ragionamento è il modo in cui l'agente prende le sue decisioni e mette in pratica le azioni. Esso è composto di otto fasi: le prime tre sono quelle che riguardano l'aggiornamento dei belief relativi al mondo e agli altri agenti, mentre altre descrivono la selezione di un evento che permette l'esecuzione di un'intenzione dell'agente.

%/---AGGIORNAMENTO BELIEF BASE---/
\subsubsection{a. Percezione ambiente}
La percezione effettuata dall'agente all'interno del ciclo di ragionamento è utilizzata per poter aggiornare il suo stato. L'agente interroga dei componenti capaci di rilevare i cambiamenti nell'ambiente \cite{AgentSpeakInJason} e di emettere dati consultabili utilizzando opportune interfacce.

\subsubsection{b. Aggiornamento beliefs}
Ottenuta la lista delle percezioni è necessario aggiornare la `belief base'. Ogni percezione non ancora presente nel set viene aggiunta e al contrario quelle presenti nel set e che non sono nella lista delle percezioni vengono rimosse \cite{AgentSpeakInJason}.
Ogni cambiamento effettato nella `belief base' produce un evento: quelli generati da percezioni dell'ambiente sono detti eventi esterni; quelli interni, rispetto agli altri, hanno associata un'intenzione.

\subsubsection{c. Ricezione e selezione messaggi}
L'altra sorgente di informazioni per un agente sono gli altri agenti presenti nel sistema. L'interprete controlla i messaggi diretti all'agente e li rende a lui disponibili \cite{AgentSpeakInJason}: ad ogni iterazione del ciclo può essere processato solo un messaggio. Inoltre,può essere assegnata una priotità ai messaggi in coda definendo una funzione di prelazione per l'agente.
\\
Prima di essere processati i messaggi passano all'interno di una funzione di selezione che definisce quali messaggi possano essere accettati dall'agente \cite{AgentSpeakInJason}. Questa funzione può essere implementata ad esempio per far ricevere solo i messaggi di un certo agente.

%/---SELEZIONE EVENTO E ESECUZIONE INTENZIONE---/
\subsubsection{d. Selezione evento}
Gli eventi rappresentano la percezione del cambiamento nell'ambiente o dello stato interno dell'agente \cite{AgentSpeakInJason}, come il goal. Ci possono essere vari eventi in attesa ma in ogni ciclo di ragionamento può esserne gestito uno solo, il quale viene scelto dalla funzione di selezione degli eventi che ne seleziona uno dalla lista di quelli in attesa. Se la lista di eventi fosse vuota si passa direttamente alla penultima fase del ciclo di ragionamento \cite{AgentSpeakInJason}, ovvero la selezione di un'intenzione.

\subsubsection{e. Recupero piani rilevanti}
Una volta selezionato l'evento è necessario trovare un piano che permetta all'agente di agire per gestirlo. Per fare ciò viene recuperata dalla `Plan Library' la lista dei piani rilevanti, verificando quali possano essere unificati con l'evento selezionato \cite{AgentSpeakInJason}. L'unificazione è il confronto relativo a predicati e termini. Al termine di questa fase si otterrà un set di piani rilevanti per l'evento selezionato che verrà raffinato successivamente.

\subsubsection{f. Selezione piano appplicabile}
Ogni piano ha un contesto che definisce con quali informazioni dell'agente può essere usato.
Per piano applicabile si intendeno quelli che, in relazione allo stato dell'agente, possono avere una possibilità di successo. Viene quindi controllato che il contesto sia una conseguenza logica della `belief base' dell'agente \cite{AgentSpeakInJason}. Vi possono anche essere più piani in grado di gestire un evento ma l'agente deve selezionarne uno solo ed impegnarsi ad eseguirlo.
\\
La selezione viene fatta tramite un'apposita funzione che inoltre tiene conto dell'ordinamento dei piani in base alla loro posizione nel codice sorgente oppure dell'ordine di inserimento. Quando un piano è scelto, viene creata un'istanza di quel piano che viene inserita nel set delle intenzioni \cite{AgentSpeakInJason}: sarà l'istanza ad essere manipolata dall'interprete e non il piano nella libreria.

Ci sono due possibili modalità per la creazione di un'intenzione e dipende dal fatto che l'evento selezionato sia esterno o interno \cite{AgentSpeakInJason}. Nel primo caso viene semplicemente creata l'intenzione, altrimenti viene inserita un'altra intenzione in testa a quella che ha generato l'evento, poichè è necessario eseguire fino al completamento un piano per raggiungere tale goal.

\subsubsection{g. Selezione intenzione}
A questo punto, se erano presenti eventi da gestire, è stata aggiunta un'altra intenzione nello stack. Un agente ha tipicamente più di un'intenzione nel set delle intenzioni che potrebbe essere eseguita, ognuna delle quali rappresenta un diverso punto di attenzione \cite{AgentSpeakInJason}. Ad ogni ciclo di ragionamento avviene l'esecuzione di una sola intenzione, la cui scelta è importante per come l'agente opererà nell'ambiente.

\subsubsection{h. Esecuzione intenzione}
L'intenzione, scelta nello step precedente, non è altro che il corpo di un piano formato da una sequenza di istruzioni, ognuna delle quali, una volta eseguita, viene rimossa dall'istanza del piano. Terminata l'esecuzione un'intenzione, quest'ultima viene restituita al set delle intenzioni a meno che non debba aspettare un messaggio o un feedback dell'azione \cite{AgentSpeakInJason}: in questo caso viene memorizzata in una struttura e restituita una volta ricevuta la risposta.
Se un'intenzione è sospesa non può essere selezionata per l'esecuzione nel ciclo di ragionamento.

\subsubsection{Scambio di messaggi}
Lo scambio di messaggi è la comunicazione standard che avviene tra agenti per comunicare tra loro e operare in base al contenuto ricevuto.
La comunicazione definita da AgentSpeak utilizza tre parti. La prima è la coda dei messaggi in input, ovvero una lista contenente tutti i messaggi che il sistema o interprete riceve e che sono destinati all'agente. La seconda è la coda dei messaggi di output che si allunga ogni volta che l'agente vuole inviare un messaggio ad un altro agente. L'ultima è una struttura all'interno della quale vengono memorizzate le intenzioni che sono sospese dall'esecuzione poichè aspettano una risposta dal canale di comunicazione dei messaggi.
\\
L'interprete è il mezzo per il quale i messaggi trasmessi. Esso infatti ha il compito di recuperare tutti i messaggi nella coda in uscita di ogni agente e successivamente recapitarli. Per la consegna viene recuperato l'agente destinatario di ogni messaggio e poi quest'ultimo viene posizionato nella coda di quelli in input dell'agente, in modo tale che possa recuperarne il contenuto al prossimo ciclo di ragionamento.


%----------------------------



% crea il CAPITOLO
\chapter{Percorso di approfondimento}
% imposta l'intestazione di pagina
\lhead[\fancyplain{}{\bfseries\thepage}]{\fancyplain{}{\bfseries\rightmark}}
% mette i numeri arabi
%\pagenumbering{arabic}

In questo capitolo sono descritti i processi evolutivi di apprendimento dei due strumenti principali per lo sviluppo del lavoro di tesi, tuProlog e Alchemist. Inoltre viene descritto anche come è stato progettato il processo di integrazione tra il meta-modello di Alchemist e il modello ad agenti.

%----------------------------
\section{Approfondimento di tuProlog}
Per quello che riguarda lo studio di tuProlog per prima cosa sono stati recuperati i materiali forniti durante le lezioni e i laboratori, dai quali sono stati estratti i concetti principali, non solo di tuProlog ma anche di Prolog.
Le informazioni recuperate sono state poi raffrontate con la versione `3.0.0' del documento `Manuale tuProlog' attraverso il quale è stato possibile approfondire ulteriori aspetti di tuProlog.
\\
Sono state quindi estratte le informazioni relativamente ai predicati (libreria, teoria, built-in), alle entità della sintassi (atomi, variabili, operatori, \ldots) e configurazione del motore. Inoltre, si sono apprese ulteriori nozioni riguardo agli unificatori e agli operatori di conoscenza, ad esempio per inserire o recuperare informazioni dalle teorie o librerie.
\\
Inoltre, è stato molto utile approfondire lo studio delle librerie, alcune delle quali sono state poi utilizzate per la gestione della teoria dell'agente. In particolare, si è compreso come gestire la creazione e la decomposizione di termini e strutture e il recupero e la gestione delle clausole (utilizzate poi per la gestione dei fatti e delle regole presenti nelle teorie).

Una parte certamente di risalto è da attribuire ad una funzionalità di tuProlog, già citata in precedenza, che consente una gestione dell'integrazione fra i due strumenti selezionati. Questo è possibile poichè tuProlog supporta la programmazione multi-paradigma e, attraverso l'apposita libreria OOLibrary, consente di manipolare oggetti Java anche tramite una teoria.
\\
La funzionalità specifica che verrà poi utilizzata nella tesi è la possibilità di registrare all'interno di variabili della teoria tuProlog oggetti istanziati lato Java. In questo modo sarà possibile invocare funzioni o manipolare le proprietà di un certo oggetto in relazione all'implementazione presente nella teoria caricata nel motore tuProlog.
Questo apre ad una serie di scenari che possono facilitare il design dell'architettura ed inoltre agevolare l'implementazione del modello ad agenti.

%----------------------------
\section{Approfondimento di Alchemist}
Il processo di apprendimento di Alchemist è stato un po' più lungo rispetto a quello di tuProlog per via del numero di parti da cui è composto.

La prima fase è stata dedicata allo studio del meta-modello sul quale questo ambiente di simulazione si sviluppa; sono state analizzate le varie entità per capire il loro ruolo e il loro comportamento durante l'esecuzione della simulazione.

Dopo il modello ci si è dedicati ad un'analisi relativa al materiale già presente all'interno di Alchemist. Partendo dalla documentazione, sono state analizzate le interfacce di ogni entità per approfondire le caratteristiche e studiarne le peculiarità da poter sfruttare in fase di design per il progetto di tesi.
\\
Successivamente è stata effettuate un'indagine sulle incarnazioni già presenti all'interno di Alchemist, in modo da capire in quali modi è stato utilizzato questo simulatore e in che modo sfruttare a pieno il modello.
\\
Proprio in questa fase, si è iniziato a pensare ai possibili punti di contatto tra il modello ad agenti e il meta-modello proposto da Alchemist. La descrizione del ragionamento e delle conclusioni è riportata nella sezione \ref{sctn:mapping}.

\subsection{Mapping modelli}\label{sctn:mapping}
\`E necessario capire quale risulta il migliore modo, in termini di performace e espressività, per unire i due modelli.
In questa fase si vuole quindi pensare come realizzare sul meta-modello fornito da Alchemist il modello ad agenti cercando eventuali incongruenze o opportunità per massimizzare il risultato.

Si è partiti analizzando le entità del meta-modello e per ognuna è stato posto l'interrogativo sul fatto che potesse essere un'agente.
Fin da subito sono state ritenute inadatte l'Environment e la Molecola: il primo perchè è esso stesso lo spazio e non avrebbe potuto rappresentare lo spazio degli agenti, mentre le molecole perchè forniscono un livello di dettaglio troppo elevato e non hanno una struttura che consente di contenere lo stato dell'agente.

Le entità rimaste da analizzare sono quindi il Nodo e la Reazione. Mappando il Nodo come agente ne deriva che l'Environment corrisponderà allo spazio degli agenti mentre, all'interno dell'agente, le Molecole e le Concentrazioni potranno essere utilizzate per gestire la `belief base' e le reazioni saranno riferite ai piani, utilizzando le Condizioni come clausola per scatenare le Azioni. Questo tipo di mapping consente di realizzare simulazioni di sistemi non complessi, in cui agenti allo stesso livello operano e comunicano tra loro.

Posizionando l'agente nella Reazione, quindi più internamente rispetto al precedente mapping, il Nodo sarà quindi un contenitore di agenti e l'Environment lo spazio nel quale si muovono i gruppi di agenti. Ogni agente avrà il riferimento ad una Condizione e ad una Azione: quest'ultima conterrà il ciclo di ragionamento dell'agente mentre la Condizione, sempre vera, ne determinerà la clausola di esecuzione. Utilizzando questa seconda ipotesi sarà possibile realizzare simulazioni di sistemi complessi, nei quali dei nodi, che potrebbero essere dispositivi mobili (ad esempio cellulari), si muovo nello spazio ed ognuno dei quali al suo interno contiene un gruppo di agenti che possono interagire sia internamente che esternamente.

Dopo aver analizzato le due possibili alternative presentate, è stato scelto il mapping in cui l'agente è posizionato nella Reazione poichè permette una maggiore espressività e un'apertura verso più scenari applicativi.

\subsection{Configurazione di una simulazione}
Successivamente ci si è dedicata alla comprensione della struttura della configurazione di una simulazione. Per poter scrivere una simulazione è necessario per prima cosa imparare le nozioni base di YAML, poichè il documento che il simulatore si aspetta in input è una mappa definita proprio tramite questo linguaggio.
La struttura di una simulazione contiene una serie di keyword, che possono essere obbligatorie o opzionali, tra cui:
\begin{itemize}
\item incarnation, per definire l'incarnazione
\item environment, per definire l'ambiente
\item network-model, per definire la linking-rule
\item displacement, per definire la disposizione e la tipologia di nodi e reazioni
\end{itemize}
le quali, in base al tipo, contengono un valore oppure una mappa innestata nella quale specificano altri parametri.

Per effettuare i test delle problematiche analizzate durante lo svolgimento dell'attività propedeutica sono state scritte simulazioni molto basilari, dove ad esempio era presente un nodo con uno solo agente (tranne nel caso dello scambio di messaggi).

Un esempio di configurazione è quello presentato nel Codice sorgente \ref{lst:SimulationExample}.

\lstset{
  basicstyle=\ttfamily,
  captionpos=b,
  frame=tb,
  numbers=left
}

\medskip
\begin{lstlisting}[firstnumber=1,label={lst:SimulationExample},caption={Esempio configurazione di una simulazione}]
incarnation: agent

network-model:
  type: ConnectWithinDistance
  parameters: [10]

displacements:
  - in: {type: Circle, parameters: [1,0,0,2]}
    programs:
      -
        - time-distribution: 1
          program: "some_agent"
\end{lstlisting}

In questo caso si vuole eseguire una simulazione che utilizzi l'incarnazione ad agenti. L'environment non è espresso e sarà considerato quello di default ( ovvero Continuous2DEnvironment), mentre è specificata la keyword `network-model' utilizzata per la linking-rule: sfruttando la classe `ConnectWithinDistance' e passandogli un certo parametro si vuole che i nodi, la cui distanza è minore di quella indicata dal parametro, appartengano allo stesso vicinato.
\\
Nell'ultima parte viene invece specificata la disposizione dei nodi e delle reazioni o azioni in essi contenute. Nel caso mostrato, verrà creato, all'interno di un cerchio di centro (0,0) e di raggio 2, un solo nodo al cui interno sarà presente una reazione, identificata opportunamente dal parametro `some\_agent', la quale avrà come distribuzione temporale 1, ovvero verrà eseguita una volta ad ogni trigger da parte dello scheduler di Alchemist.

\section{Integrazione di modelli e strumenti}
Il passo successivo è stato quello di iniziare a prendere confidenza con il simulatore provando ad implementare alcuni problemi basilari che sarebbero tornati utili in futuro. Alcuni dei problemi affrontati sono:
\begin{enumerate}
\item agente che effettua una computazione elementare
\item agente che effettua una computazione condizionata
\item agente che è in grado di spostarsi nello spazio
\item scambio di messaggi tra agenti (in questo punto è stato unito tuProlog)
\end{enumerate}

Nel fronteggiare i vari quesiti sono state prodotte implementazioni di piccole parti che sono servite come punto di partenza per la realizzazione del progetto di tesi.

Si è iniziato implementando l'incarnazione attraverso la creazione del nodo e di una reazione, la quale al suo interno contiene un'azione. Quest'ultima è l'entità che viene innescata, se le condizioni si verificano, e attraverso il metodo $execute$ (invocato dal simulatore) esegue una serie operazioni definite nell'implementazione dell'azione.

I primi due quesiti sono risultati complessi non tanto per la loro natura ma per lo scoglio iniziale nell'approccio ad Alchemist. Successivamente si è passati allo spostamento del nodo, sul quale è stato effettuato un ragionamento per permettere al nodo di spostarsi in ogni direzione e della distanza esatta percorsa nel tempo della simulazione. La soluzione prevede l'utilizzo di variabili per velocità, direzione e tau (tempo della simulazione dell'ultimo aggiornamento) con i quali è costruito un cerchio che ha come centro l'ultima posizione del nodo e come raggio la distanza calcolata con tempo (differenza tra tau attuale e quello dell'ultimo aggiornamento) e velocità. Sulla circonferenza, in base all'angolo definito dalla direzione, verrà preso il punto della nuova posizione del nodo.
\\
Dopo aver realizzato i primi tre quesiti descritti nel precedente elenco, si è passati alla risoluzione dell'ultima problematica per avere conoscenza dell'intera situazione, consentendo di avere cognizione dell'intero problema e del supporto che possono fornire gli strumenti descritti in questo documento. Per la realizzazione si è utilizzato il motore tuProlog all'interno della quale è stata caricata una semplice teoria che mandava indietro il messaggio ricevuto (l'esempio realizzato è stato il Ping Pong).

La parte di integrazione di tuProlog è stata effettuata importanto la libreria `alice.tuprolog' in Alchemist e poi utilizzandone il motore, sia per caricare le teorie sia per l'interrogazione di quest'ultime.
La parte di registrazione di oggetti Java all'interno di variabili tuProlog verrà inserita direttamente durante l'implementazione del progetto di tesi.



\chapter{Test case}\label{chap:validation}
In questo capitolo è mostato come il linguaggio e l'incarnazione realizzati sono stati applicati ad uno scenario di test noto per quanto riguarda i sistemi multi agente (MAS).
Viene per prima cosa descritto il problema e successivamente analizzati i vari ruoli all'interno della scena.
\\
La soluzione proposta è una ma non l'unica possibile. Una possibile alternativa può esser ad esempio la modifica del modo in cui le pepite sono consegnate. Trasformando il deposito in uno spazio di tuple si dovrà quindi modificare l'azione dell'agente $iSend(R,M)$ sostituendola con $writeTuple(R,T)$, dove $R$ è il destinatario, ovvero il deposito, e la pepita si trasforma da un messaggio $M$ in una tupla $T$.

\section{Descrizione problema}
Il quesito che è stato preso in esame per presentare le funzionalità del progetto sviluppato è noto come `Goldminers' ed è stato definito da Jomi H\"ubner e Rafael Bordini.

\medskip
\fbox{\parbox[t][][t]{1\textwidth}{
In questo scenario un gruppo di agenti minatori deve recuperare pepite d'oro da miniere sparse nell'ambiente e riportarle in un deposito.
}}
\medskip

Il quesito è stato quindi applicato al progetto realizzato in questo lavoro di tesi e sono stati delineati i comportamenti dei ruoli (minatore, miniera, deposito) all'interno della scena: ognuno di essi sarà una specializzazione dell'agente. All'interno di ogni nodo sarà contenuto solo un agente, oltre a quello per il movimento che è `fissato' nel nodo, per via dell'implementazione dell'interprete.
\\
Nella scena gli agenti sono posizionati casualmente all'interno di aree delimitate da cerchi, i quali sono posizionati nell'ambiente secondo le direttive della configurazione. Per ogni tipologia di ruolo è definita l'area del cerchio nel quale sono posizionati gli agenti.
\\
Qui di seguito è descritto come il problema sarà implementato all'interno dell'interprete realizzato.

I minatori conoscono la posizione del deposito: il loro compito è quello di recuperare le pepite e consegnarle al deposito.
Le miniere sono contenitori di risorse, le pepite, che sono estratte dai minatori. Il numero di pepite contenute all'interno di ogni miniera è casuale.

Quando inizia la simulazione i minatori si muovono nell'ambiente alla ricerca di una miniera. Lo spostamento dei nodi viene comandato dall'agente minatore secondo la distribuzione di Levy.
Se il minatore incontra una miniera prova a prelevare una pepita e una volta recuperata la risorsa avvia lo spostamento verso il deposito per recapitare la pepita. Dopo la consegna il minatore riparte verso la miniera da cui ha estratto la risorsa per continuare ad estrarre fino ad estinguerla.
Quando il minatore esaurisce la miniera torna allo stato iniziale, ovvero muovendosi nell'ambiente alla ricerca di nuove miniere da cui estrarre pepite.


\subsection{Tematizzazione della simulazione}
Alchemist nella finestra della simulazione permette di definire degli effetti da applicare ai nodi della simulazione per tematizzarli. La finestra attraverso la quale è possibile definire lo stile è mostrata in Figura \ref{fig:tematizzazioneSimulazione}.

%crea l'ambiente figura;
\begin{figure}[h] % [h] sta per here, cioè la figura va qui
\begin{center} % centra nel mezzo della pagina la figura
\includegraphics[width=6cm]{images/tematizzazioneSimulazione.png} % inserisce una figura larga 12.5cm
% inserisce la legenda ed etichetta la figura con \label{fig:prima}
\caption[Tematizzazione della simulazione]{Tematizzazione della simulazione} \label{fig:tematizzazioneSimulazione}
\end{center}
\end{figure}

Dall'immagine si può osservare che l'effetto può essere composto da molteplici fattori.
Per quanto riguarda la tematizzazione scelta per questa simulazione sono stati utilizzati i controlli presenti nella prima metà della finestra.
\newline

Sono state definite delle stringhe identificative per le molecole che si riferiscono univocamente ai vari ruoli nella scena: minatore (miner), miniera (goldmine), deposito (deposit), pepita (nugget).
Per stringa è stato creato un relativo stile definito da un colore (tramite il modello RGBA), un fattore di scala e una dimensione per rappresentare il nodo.
\\
L'associazione tra ogni effetto e la rispettiva molecola viene realizzato spuntando la casella di controllo relativa a `Draw only nodes containing a molecule' ed inserendo nella casella di testo sottostante la stringa identificativa della molecola che si vuole abbinare.
\newline

Nell'implementazione dell'interprete sono opportunamente create le molecole utilizzando la classe SimpleMolecule, passando come parametro identificativo le stringhe descritte poco sopra. Gli oggetti istanziati sono poi inseriti all'interno del nodo.

Una volta creato lo stile desiderato è possibile salvarlo (viene generato un file con estensione .aes) in modo da poterlo riutilizzare.
All'avvio di una simulazione di default non è presente nessuna tematizzazione però è possibile caricare un file salvato in precedenza.
La tematizzazione non è necessaria al fine dell'esecuzione della simulazione ma è molto utile per poter comprendere al meglio il comportamento degli agenti nell'avanzamento della simulazione.


\section{Implementazione agenti}
L'agente, come già accennato in precedenza, è composto e definito tramite due parti:
\begin{itemize}
\item la \textbf{teoria} che definisce come e a quali eventi reagire in base ad un contesto;
\item la \textbf{classe} che definisce in che modo i comportamenti sono trasmessi dall'interprete alla teoria.
\end{itemize}
In questa sezione sono descritte le teorie e le classi degli agenti che sono stati implementati per la realizzazione dello scenario di test.

\subsection{Miniera - Goldmine}
La miniera è un'entità che è posizionata in modo casuale all'interno dell'ambiente, che mantiene la sua posizione nel tempo e che contiene delle risorse.
Data questa descrizione la sua realizzazione è stata subito associata agli spazi di tuple. Si è quindi implementata la classe Goldmine estendendo la classe AbstractSpatialTuple.
\\
Per iniziare è stata scritta la teoria dell'agente in cui è definito il suo comportamento e solo successivamente è stata implementata la classe all'interno dell'interprete e le relative funzionalità.

\subsubsection{Teoria tuProlog miniera}
La teoria di questo agente è poco complessa poichè definisce solamente il caricamento di un numero casuale di risorse (ovvero le pepite) sotto forma di tuple all'interno dello spazio di tuple.
\lstset{
  basicstyle=\ttfamily,
  captionpos=b,
  numbers=none,
  frame=tb,
  stringstyle=\color{black}
}
\medskip
\begin{lstlisting}[firstnumber=1,label={lst:Goldmine},caption={Teoria miniera}]
init :-
    agent <- generateNextRandom returns RAND,
    NUGGETS is RAND * 1.0,
    loadNuggets(NUGGETS).

loadNuggets(NUGGETS) :-
    NUGGETS > 0,
    assertz(nugget),
    N is NUGGETS - 1,
    loadNuggets(N).

loadNuggets(NUGGETS) :-
    NUGGETS < 0,
    agent <- setConcentration.
\end{lstlisting}

Nel corpo della regola $init$ viene inizialmente recuperato un numero random, attraverso l'invocazione del metodo $generateNextRandom()$ definito nell'agente, che rappresenta le pepite da posizionare all'interno della miniera.
Il caricamento delle risorse viene fatto tramite l'invocazione della regola $loadNuggets(NUGGETS)$ con la quale sono inseriti i fatti nello spazio di tuple: se le risorse sono state tutte caricate viene invocato il metodo $setConcentration()$,definito appositamente all'interno dell'agente, che viene utilizzato per impostare le molecole per tematizzare graficamente lo spazio di tuple nella simulazione.

\subsubsection{Implementazione classe miniera}
Per quanto riguarda la classe della miniera si è deciso, come detto poco sopra, di utilizzare lo spazio di tuple. La classe astratta AbstractSpatialTuple definisce al suo interno la struttura base dello spazio di tuple e le funzioni per eseguire i suoi comportamenti caratteristici $in$, $rd$ e $out$, rispettivamente trasformati nelle azioni $writeTuple$, $readTuple$, $takeTuple$.
Per definire la miniera è stata quindi definita la classe Goldmine che estende la classe astratta implementando quindi un semplice spazio di tuple.
Oltre ad essere uno spazio di tuple, la classe però implementa anche la super classe astratta AbstractAgent che fa essere Goldmine anche un agente: è quindi necessario implementare all'interno della classe anche i metodi $execute()$ e $cloneAction(node, reaction)$. Il primo di questi due metodi è quello più importante, poichè definisce in che modo lo spazio di tuple opererà nel suo ciclo di ragionamento permettendo a quest'ultimo di reagire alle richieste ricevute.
\\
Per la tematizzazione all'interno della simulazione è stato implementato il metodo $setConcentration()$. Questo metodo, come già accennato nella sezione precedente, viene utilizzato per inserire un oggetto molecola nel nodo, la quale è utilizzata per la tematizzazione della simulazione. La molecola viene creata ed inserita con le istruzioni presenti nel Codice sorgente \ref{lst:CreazioneInserimentoMolecola} solo dopo aver verificato che nella teoria dell'agente sono effettivamente presenti delle risorse.
\lstset{
  numberstyle=\footnotesize\color{black},
  basicstyle=\ttfamily,
  breakatwhitespace=true,
  breaklines=true,
  captionpos=b,
  keepspaces=true,
  numbers=left,
  numbersep=7pt,
  showspaces=false,
  showstringspaces=false,
  showtabs=false,
  tabsize=2,
  frame=tb,
  language=Java,
  commentstyle=\color{gray},
  keywordstyle=\color{blue},
  stringstyle=\color{red}
}
\medskip
\begin{lstlisting}[firstnumber=1,label={lst:CreazioneInserimentoMolecola},caption={Creazione e inserimento molecola}]
SimpleMolecule sm = new SimpleMolecule("nugget");
this.getNode().setConcentration(sm, 0);
\end{lstlisting}
La stringa `nugget' è quella che deve essere inserita nella casella di testo a fianco dell'etichetta Molecule nella Figura \ref{fig:tematizzazioneSimulazione} e che deve essere abilitata cliccando la checkbox appena sopra.
\newline

Per via della necessità di tematizzare la simulazione si è dovuto sovrascrivere il metodo dello spazio di tuple relativo alla funzionalità $out$, o $takeTuple$.
Qui di seguito è descritta la funzionalità già implementata nella classe astratta. Nel paragrafo successivo viene invece riportata la modifica effettuata per definire la tematizzazione della miniera all'interno della simulazione.

Nell'implementazione della funzionalità già definita all'interno della classe astratta, la procedura è la seguente:
\begin{enumerate}
\item recupero di una tupla che corrisponda al template tramite il predicato $retract$;
\item interrogazione per ottenere dal termine estratto il template popolato con i valori recuperati;
\item invio del template popolato tramite un messaggio all'agente che lo aveva richiesto.
\end{enumerate}
Nel caso l'operazione di recupero della tupla non vada a successo quella richiesta viene aggiunta tra quelle in attesa, le quali, se previsto nel ciclo di ragionamento, periodicamente vengono verificate nuovamente per poter essere completate.
Nel caso in cui la richiesta del template è andata a successo e questa è presente nell'elenco di quelle in attesa allora la stessa viene rimossa prima di procedere con l'operazione indicata nell'elenco con il numero 2.

La modifica effettuata è relativa solamente alla tematizzazione, cioè all'inserimento o alla rimozione di un oggetto molecola, e quindi la funzionalità principale rimane quella appena descritta.
Il codice aggiunto è stato inserito prima di effettuare l'operazione di invio del messaggio all'agente. L'aggiunta consiste in un'interrogazione per verificare se dopo aver estratto una risorsa sono presenti ancora pepite, presenti nella teoria sotto forma di fatti, all'interno della miniera. In caso affermativo viene lasciata la molecola che era stata impostata precedentemente invocando dalla teoria la funzionalità $setConcentration()$. Diversamente, se non sono più presenti risorse nella miniera la molecola viene rimossa e, se la tematizzazione della simulazione è opportunamente configurata, il cambiamento porterà ad una modifica dello stile del nodo.

\subsection{Minatore - Miner}
Il minatore è l'agente con il comportamento più complesso per lo scenario preso in esempio.
Il suo compito è quello di muoversi nell'ambiente cercando miniere da cui estrarre risorse da portare al deposito.
Il comportamento del minatore è stato scomposto in 4 fasi o stati:
\begin{enumerate}
\item harvesting: spostamento casuale e ricerca di risorse all'interno di miniere;
\item toDeposit: recuperata la pepita, spostamento verso il deposito per consegnarla;
\item toMine: depositata la pepita, ritorno alla miniera;
\item arrivato alla miniera torna nello stato harvesting.
\end{enumerate}

\subsubsection{Teoria tuProlog minatore}
Le fasi descritte sono state poi riportate all'interno della teoria dell'agente, mostrata nel Codice sorgente \ref{lst:Goldmine}, per definirne il comportamento in relazione agli eventi innescati dall'interprete. Le invocazioni evidenziate in blu sono relative a regole definite all'interno della teoria dell'agente e che non sono state mostrate.
\\
Le regole $ randomDirection(D)$ e $randomSpeed(S)$ restituiscono nella variabile passata, rispettivamente $D$ e $S$, il valore ottenuto dall'invocazione combinata delle funzioni definite all'interno della classe AbstractAgent, e quindi disponibile in tutte le implementazioni di un agente, che generano un valore random e lo applicano alla distribuzione di Levy.
\\
La regola $handlePosition$ si occupa, in modo molto simile a quello che viene fatto nella regola $init$, di generare una nuova velocità e un delta per la direzione da impostare nel nodo, mentre $changeDirection(X,Y)$ modifica solamente la direzione calcolando il valore corretto per raggiungere il punto (X,Y) data la posizione corrente.
\\
La regola $checkMineDistance$ definisce la guardia che verifica che l'agente, dopo aver consegnato la pepita al deposito e essersi diretto alla miniera, è effettivamente arrivato a destinazione e può tornare nello stato `harvesting'.

\lstset{
  basicstyle=\ttfamily,
  captionpos=b,
  numbers=none,
  frame=tb,
  stringstyle=\color{black},
  morekeywords={randomDirection,randomSpeed,handlePosition,changeDirection,checkMineDistance}
}
\medskip
\begin{lstlisting}[firstnumber=1,label={lst:Miner},caption={Teoria minatore}]
init :-
    addBelief(deposit(2,2)),
    addBelief(harvesting),
    randomDirection(D),
    randomSpeed(S),
    node <- changeNodeSpeed(S),
    node <- changeDirectionAngle(D).

%(1)
'<-'(onAddBelief(position(X,Y)), [belief(harvesting)], [handlePosition, takeTuple(nugget)]).

%(4)
'<-'(onAddBelief(position(X,Y)), [checkMineDistance(X,Y,MX,MY)], [removeBelief(mine(_,_)), removeBelief(toMine), addBelief(harvesting), changeDirection(MX,MY)]).

%(2)
'<-'(onResponseMessage(msg(nugget,X,Y)), [removeBelief(harvesting)], [addBelief(toDeposit), addBelief(mine(X,Y)), belief(deposit(DX,DY)), changeDirection(DX,DY)]).

%(3)
'<-'(onAddBelief(distance(deposit,ND)), [removeBelief(toDeposit)], [iSend(deposit,nugget), addBelief(toMine), belief(mine(X,Y)), changeDirection(X,Y)]).
\end{lstlisting}
Nella fase di configurazione iniziale, regola $init$, vengono impostati all'interno della teoria del minatore il belief relativo alla posizione del deposito e quello relativo allo stato iniziale dell'agente, `harvesting'. Successivamente sono generati i valori per direzione e velocità che poi sono impostati nel nodo.
\\
Dopo la regola $init$ sono descritti i fatti che definiscono il comportamento dell'agente, mostrato ad inizio sezione, e che gli permettono di reagire al verificarsi di opportuni eventi.
I fatti seguono la seguente struttura $'<-'(EVENT, GUARD, BODY).$, dove $EVENT$ è l'evento al quale l'agente vuole reagire, $GUARD$ identifica il contesto o la condizione per cui le azioni contenute nel $BODY$ possano essere eseguite.
\\
I fatti identificati dalle fasi 1 e 4 agiscono entrambi in relazione all'evento di aggiornamento della posizione. Il contesto della fase 1 è relativo allo stato `harvesting' mentre quello della fase 4 è la vicinanza del nodo che contiene l'agente rispetto alla miniera che deve raggiungere.
Gli altri due contesti, relazionati ad eventi quali la ricezione di un messaggio dallo spazio di tuple e l'aggiornamento della distanza dal nodo contenente l'agente deposito, sono associati alla rimozione di un certo stato che quindi risulta bloccante finchè lo stato da rimuovere non è presente all'interno della `belief base' dell'agente.
\\
Tra le azioni definite nei corpi dei vari fatti una spiegazione la merita l'utilizzo del belief $mine(X,Y)$. Questo viene utilizzato per salvare la posizione della miniera da cui è stata recuperata l'ultima risorsa, prima di andare al deposito per consegnarla, in modo tale da poter tornare direttamente alla miniera, una volta depositata la pepita, per continuare ad estrarre risorse risultando più veloce ed efficace.

\subsubsection{Implementazione classe minatore}
Per quanto riguarda la parte lato interprete relativa all'agente minatore è stata utilizzata la classe SimpleAgent che estende da AbstractAgent. Questa classe rappresenta un esempio di come può essere immediata l'implementazione di un agente, che racchiude tutte le funzionalità principali, a partire da quello che è già stato prodotto e mostrato precedentemente in questo documento.
\medskip
\begin{lstlisting}[firstnumber=1,label={lst:SimpleAgentReasoningCycle},caption={Ciclo di ragionamento per l'agente completo}]
@Override
public void execute() {
    if (this.isInitialized()) {

        //Agent's reasoning cycle

        this.beliefBaseChanges();

        this.readMessage();

        // SpatialTuples extension
        this.retrieveTuples();

        this.executeIntention();
    } else {
        this.initializeAgent();

        this.initReasoning();
    }
}
\end{lstlisting}
Per implementare la classe SimpleAgent (denominata in questo modo perchè implementa un agente con tutte le funzionalità principali) è stato sufficiente implementare il costruttore, richiamando quello di AbstractAgent, e i due metodi $cloneAction(node, reaction)$ e $execute()$.
Nel Codice sorgente \ref{lst:SimpleAgentReasoningCycle} è mostrato come è stato definito il ciclo di ragionamento.
Come si può notare è stata utilizzata una condizione, che utilizza un flag definito nella classe astratta, per determinare se l'agente è stato inizializzato, ovvero se ha completato la configurazione iniziale.
\\
Nel primo ciclo di ragionamento avviene appunto la configurazione iniziale. La prima funzione che viene chiamata è $initializeAgent()$ che termina la configurazione utilizzata dall'interprete, e all'interno della quale viene modificato il flag citato in precedenza, e poi viene invocata $initReasoning()$ che esegue quanto previsto dalla regola $init$ se presente nella teoria dell'agente.
\\
Dal ciclo di ragionamento successivo vengono eseguite le funzioni per:
\begin{enumerate}
\item controllare le modifiche alla `belief base';
\item leggere i messaggi ricevuti;
\item inviare le richieste agli spazi di tuple;
\item eseguire un'intenzione.
\end{enumerate}
La funzione $beliefBaseChanges()$ recupera tutte le modifiche effettuate alla `belief base' e, per ognuna per cui esiste un comportamento nella teoria, genera un'intenzione inserita opportunamente nella teoria. In modo simile, $readMessage()$ recupera, se presente, un messaggio dalla lista di quelli in entrata e, anche in questo caso, se nella teoria è previsto un comportamento viene generata un'intenzione.
\\
La funzione $retrieveTuples()$ recupera ed invia immediatamente tutte le richieste agli spazi di tuple opportuni.
Per concludere viene invocata la funzione $executeIntention()$ che sceglie un'intenzione tra quelle presenti all'interno dell'agente (il metodo di selezione è Round-Robin), la esegue e poi la inserisce in coda alla lista delle intenzioni.

\subsection{Deposito - Deposit}
Per quanto riguarda il deposito, il suo compito è semplicemente quello di raccogliere le risorse consegnate dai minatori.
Nella soluzione proposta il deposito è stato implementato da un agente ma, come detto nell'introduzione del capitolo, è possibile realizzare simulazioni con un'implementazione diversa.

\subsubsection{Teoria tuProlog deposito}
Per quanto riguarda la teoria del deposito non è necessario definire alcun piano poichè esso non ha compiti se non quello di accettare le risorse ricevute. Si è quindi definita una teoria contenente una regola $init$ vuota (ovvero che ha come unica istruzione del corpo `true').

\subsubsection{Implementazione classe deposito}
La classe utilizzata per rivestire questo ruolo all'interno della simulazione è anche in questo caso la classe SimpleAgent poichè è un'implementazione di un agente completo già pronta per essere utilizzata.
\\
Le funzionalità disponibili sono sicuramente maggiori rispetto a quelle che sono effettivamente utilizzate ma, data la struttura già fornita della classe astratta, questo non impatta sull'efficienza e rimane molto semplice da utilizzare.

\section{Configurazione della simulazione}
Per definire la simulazione si deve scrivere un'opportuna configurazione nella quale sono descritti, attraverso le keyword, quali oggetti utilizzare nella simulazione in relazione al modello creato. Nel Codice sorgente \ref{lst:GoldminersSimulation} sono mostrate e descritte in dettaglio le varie parti che compongono la configurazione e che sono state precedentemente spiegate nella sezione \ref{sctn:ScrivereUnaSimulazione}.

\medskip
\begin{lstlisting}[firstnumber=1,label={lst:GoldminersSimulation},caption={Configurazione simulazione Goldminers}]
incarnation: agent

network-model:
  type: ConnectWithinDistance
  parameters: [2]

displacements:
  - in: {type: Circle, parameters: [2,2,2,0.2]}
    programs:
      -
        - time-distribution: 9
          program: "miner"

  - in: {type: Circle, parameters: [1,2,2,0.2]}
    programs:
      -
        - time-distribution: 9
          program: "deposit"

  - in: {type: Circle, parameters: [1,2,2,0.2]}
    programs:
      -
        - time-distribution: 9
          program: "postman"

  - in: {type: Circle, parameters: [10,2,2,5]}
    programs:
      -
        - time-distribution: 9
          program: "goldmine"
\end{lstlisting}

La prima cosa che si può notare nella configurazione è la specifica dell'incarnazione utilizzata `agent', ovvero quella definita per questo progetto.

Il secondo parametro impostato in configurazione è il `network-model', il parametro che definisce la `linking-rule', che descrive in che modo ogni nodo presente all'interno dell'ambiente è collegato con gli altri. Attraverso questa regola cambia il numero di nodi presenti nel vicinato di ogni nodo. Nel caso specifico, è stato scelto di utilizzare una classe che utilizza la relazione di distanza per collegare i vari nodi ed alla quale è stato passato come parametro il valore 2; per ogni nodo, sono considerati suoi vicini tutti i nodi che ad ogni istante della simulazione siano posizionati all'interno di un cerchio di dimensione 2 il cui centro è il nodo stesso.

Il terzo ed ultimo parametro è `displacements' che definisce una lista con la disposizione dei nodi ed il loro contenuto.
\\
La lista si compone di due keyword: $in$ e $programs$. La prima richiede un oggetto che definisce il tipo della geometria e i parametri per costruirla, mentre la seconda utilizza un'ulteriore lista all'interno della quale sono definite le reazioni associate ad ogni nodo.

Per tutti i ruoli della scena è stato utilizzato il cerchio come geometria per posizionare i nodi, passando per ognuno dimensioni e quantità di nodi differenti.
\\
Attraverso la keyword $program$ si definiscono le reazioni che saranno inserite nei nodi: come si può notare dalla configurazione, per ogni nodo è stata definita una sola reazione.
Per definire una reazione sono stati utilizzati una distribuzione temporale ed una stringa; quest'ultima sarà poi utilizzata durante la creazione di istanze della classe AgentReaction e con la quale verrà istanziata all'interno della reazione la classe dell'agente relativa.
\\
Come parametro della distribuzione temporale è stato scelto 9: questo valore ha consentito, anche in base al valore di distribuzione temporale associato all'azione di spostamento posizionata in maniera intrinseca nel nodo, di avere un comportamento che è sembrato molto fluido.


\section{Metriche di progetto}
Le metriche di progetto rappresentano un insieme di indicatori per tenere sotto controllo e prevedere l'andamento delle principali variabili del progetto (costi, tempi, qualità, risorse).
Le metriche includono solitamente una serie di indicatori standard ma ne possono essere aggiunti altri \textit{ad hoc} in relazione al progetto.
\\
Con le metriche è possibile quantificare in modo obiettivo le performance del progetto misurando gli indicatori predisposti. L'uso principale delle metriche è quello di misurare l'avanzamento del progetto. Il loro utilizzo permette di identificare i problemi di costo/schedulazione prima che diventino criticità e aiutare a focalizzarsi sul completamento delle attività.

\subsection{Metriche software del progetto}
Con le metriche di progetto si misura il codice, quindi è possibile analizzare solamente la parte del lavoro di tesi relativa all'implementazione dell'interprete del linguaggio all'interno del simulatore Alchemist.
La figura \ref{fig:codeMetrics} è stata ottenuta tramite l'utilizzo del plugin CodeMR Metrics disponibile nell'IDE IntelliJ IDEA. Inoltre, è stato utilizzato anche il plugin Statistics per avere un dettaglio maggiore per quanto riguarda le SLOC.
%crea l'ambiente figura;
\begin{figure}[h] % [h] sta per here, cioè la figura va qui
\begin{center} % centra nel mezzo della pagina la figura
\includegraphics[width=12cm]{images/codeMetrics.png} % inserisce una figura larga 12.5cm
% inserisce la legenda ed etichetta la figura con \label{fig:prima}
\caption[Metriche di progetto]{Metriche di progetto} \label{fig:codeMetrics}
\end{center}
\end{figure}

\medskip
Per la realizzazione dell'incarnazione ad agenti all'interno di Alchemist sono state definite dieci classi Java per implementare la struttura e mostrare come creare le diverse tipologie di agenti. Il totale delle righe presenti all'interno dei file è 2.085, di cui 1.109 sono SLOC e 751 sono di documentazione.
In questo numero non sono calcolate, come accennato precedentemente, le righe scritte per la definizione delle teorie degli agenti utilizzate per la realizzazione degli scenari di test.

La complessità ciclomatica è utilizzata per misurare la complessità di un programma misurando il numero di cammini linearmente indipendenti attraverso il grafo di controllo di flusso.
L'incarnazione realizzata all'interno di Alchemist ha un valore totale di complessità ciclomatica pari a 220, con una media di 2.10.
%crea l'ambiente figura;
\begin{figure}[h] % [h] sta per here, cioè la figura va qui
\begin{center} % centra nel mezzo della pagina la figura
\includegraphics[width=14cm]{images/complessitaCiclomatica.png} % inserisce una figura larga 12.5cm
% inserisce la legenda ed etichetta la figura con \label{fig:prima}
\caption[Complessità ciclomatica]{Complessità ciclomatica} \label{fig:cyclomaticComplexity}
\end{center}
\end{figure}

Nell'immagine \ref{fig:cyclomaticComplexity} è mostrato in dettaglio la complessità ciclomatica delle varie classi, anche quelle innestate, implementate all'interno dell'incarnazione, misurata utilizzando il plugin MetricsReloaded disponibile nell'IDE IntelliJ IDEA. Per ogni classe viene descritto il valore relativo alla complessità operazionale media (OCavg) e quello relativo alla complessità totale(WMC).
La complessità ciclomatica totale dell'incarnazione è 187, la cui media ripartita tra le viarie classi è 15.58.
Si può notare fin da subito che la classe AbstractAgent, la quale definisce tutte le funzionalità di base dell'agente, risulti essere di una complessità ben maggiore rispetto a tutte le altre classi, seguita dalla classe del nodo e poi da quelle che utilizzano il modello SpatialTuples.
Proprio tra quest'ultime, nello specifico Blackboard e Goldmine che sono due implementazioni della classe AbtractSpatialTuple, sono presenti le classi con una compkessità ciclomatica media più alta. Questa caratterizzazione può essere spiegata dal fatto che per definire gli spazi di tuple, vista l'ereditarietà dalla classe AbstractAgent, è stato necessario implementare le funzionalità di gestione dello spazio di tuple, nella classe AbstractSpatialTuple, o nuove personalizzazioni della funzionalità base, nelle implementazioni specifiche degli spazi di tuple.

Per quanto riguarda i metodi presenti all'interno delle varie classi non è mostrato il dettaglio ma sono riportati soltanto i valori totali.
La complessità ciclomatica essenziale, ev(G), è 117, la complessità di design, iv(G), è 213 mentre la complessità ciclomatica totale, v(G), è 220.



\chapter{Conclusioni}\label{chap:conclusions}
Con questo lavoro di tesi è stato definito un nuovo linguaggio per la programmazione ad agenti basato sulla struttura del modello di AgentSpeak al quale è stato integrata la gestione del ragionamento degli agenti proposta dall'interprete Jason.
%
Il linguaggio definito permette di scrivere teorie tuProlog per implementare il comportamento degli agenti sfruttando gli eventi e i predicati messi a disposizione.
Questo linguaggio non si occupa di gestire la piattaforma sulla quale opereranno gli agenti ma solo di dare una definizione degli elementi che permetteranno all'agente di essere autonomo, reattivo e proattivo.

L'interprete di questo linguaggio potrà essere implementato in qualsiasi ambiente, da quello reale a quello simulato, poichè grazie all'utilizzo della libreria tuProlog sarà possibile realizzare efficientemente la gestione delle interazioni con le primitive del nuovo linguaggio.
%
Per l'interprete quindi sarà necessario implementare un applicativo che gestisca lo scheduling degli eventi, in caso di ambiente simulato, o la comunicazione con i percettori, in caso di ambiente reali, e trasmetta le informazioni alla teoria dell'agente utilizzando il linguaggio attraverso la libreria tuProlog.

All'interno del lavoro di tesi è descritto come è stato realizzato un interprete per il linguaggio utilizzando il simulatore Alchemist.
Con questo esempio, anche attraverso alla flessibilità del meta-modello di Alchemist, si è mostrata la grande capacità di adattamento del linguaggio e la trasparenza con cui l'interprete può gestire gli eventi e gli agenti. 

\section{Vantaggi approccio scelto}
Il linguaggio è stato realizzato con l'obiettivo di fornire una nuova prospettiva più snella e dinamica che permetta di portare la potenza del modello ad agenti su varie piattaforme cercando di semplificare il lavoro di integrazione e di implementazione con questo modello.

Attraverso il linguaggio definito in questo lavoro è possibile portare il modello ad agenti da un ambiente simulato, come l'interprete che è stato realizzato all'interno di questo progetto di tesi, ad un ambiente reale, realizzando ad esempio un applicativo per dispositivi mobile ognuno dei quali esegue degli agenti, i quali comunicano con altri agenti presenti in differenti device.

%-----------------------------------------
%-----------------------------------------

\backmatter	

\bibliographystyle{plain}
\bibliography{bibtex.bib}



% non numera l'ultima pagina sinistra
%\clearpage{\pagestyle{empty}\cleardoublepage}
%\chapter{Ringraziamenti}
%\thispagestyle{empty}
%Contenuto ringraziamenti
\end{document}
