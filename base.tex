% 12pt: grandezza carattere
% a4paper: formato a4
% openright: apre i capitoli a destra
% twoside: serve per fare un
% documento fronteretro
% report: stile tesi (oppure book)
\documentclass[12pt,a4paper,openright,twoside]{report}

%libreria per scrivere in italiano
\usepackage[italian]{babel}

% libreria per accettare i caratteri
% digitati da tastiera come ?
% si può usare anche
% \usepackage[T1]{fontenc}
% però con questa libreria
% il tempo di compilazione
% aumenta
\usepackage[latin1]{inputenc}

% libreria per impostare il documento
\usepackage{fancyhdr}

% libreria per avere l'indentazione all'inizio dei capitoli, ...
\usepackage{indentfirst}

%libreria per mostrare le etichette
%\usepackage{showkeys}

% libreria per inserire grafici
\usepackage{graphicx}

% libreria per utilizzare font particolari ad esempio \textsc{}
\usepackage{newlfont}

% librerie matematiche
\usepackage{amssymb}
\usepackage{amsmath}
\usepackage{latexsym}
\usepackage{amsthm}


% example style for lstlistoflistings
% tabsize = 2, %% Sets tab space width.
% showstringspaces = false, %% Prevents space marking in strings, string is defined as the text that is generally printed directly to the console.
% numbers = left, %% Displays line numbers on the left.
% commentstyle = \color{green}, %% Sets comment color.
% keywordstyle = \color{blue}, %% Sets  keyword color.
% stringstyle = \color{red}, %% Sets  string color.
% rulecolor = \color{black}, %% Sets frame color to avoid being affected by text color.
% basicstyle = \small \ttfamily , %% Sets listing font and size.
% breaklines = true, %% Enables line breaking.
% numberstyle = \tiny,
%...................................
% backgroundcolor=\color{white},
% commentstyle=\color{green},
% keywordstyle=\color{blue},
% numberstyle=\footnotesize\color{black},
% stringstyle=\color{red},
% basicstyle=\footnotesize,
% breakatwhitespace=false,
% breaklines=true,
% captionpos=b,
% keepspaces=true,
% numbers=left,
% numbersep=10pt,
% showspaces=false,
% showstringspaces=false,
% showtabs=false,
% tabsize=2


% impostano i margini
\oddsidemargin=30pt
\evensidemargin=20pt

% serve per la sillabazione: tra parentesi vanno inserite come nell'esempio le parole
% che latex non riesce a tagliare nel modo giusto andando a capo.
\hyphenation{sil-la-ba-zio-ne pa-ren-te-si}

% comandi per l'impostazione della pagina, vedi il manuale della libreria fancyhdr per ulteriori delucidazioni
\pagestyle{fancy}\addtolength{\headwidth}{20pt}
\renewcommand{\chaptermark}[1]{\markboth{\thechapter.\ #1}{}}
\renewcommand{\sectionmark}[1]{\markright{\thesection \ #1}{}}
\rhead[\fancyplain{}{\bfseries\leftmark}]{\fancyplain{}{\bfseries\thepage}}
\cfoot{}

%comando per impostare l'interlinea
\linespread{1.3}


\begin{document}

\begin{titlepage}
% crea un ambiente libero da vincoli di margini e grandezza caratteri: si pu\`o modificare
% quello che si vuole, tanto fuori da questo ambiente tutto viene ristabilito

% elimina il numero della pagina
\thispagestyle{empty}

% imposta il margina superiore a 6.5cm
\topmargin=6.5cm

% incolonna la scrittura a destra
\raggedleft

% aumenta la grandezza del carattere a 14pt
\large

% emfatizza (corsivo) il carattere
\em

%\ldots lascia tre puntini
%\ldots

% va in una pagina nuova
\newpage

% non numera l'ultima pagina sinistra
\clearpage{\pagestyle{empty}\cleardoublepage}
\end{titlepage}

%-----------------------------------------
%-----------------------------------------
% mette i numeri arabi
\pagenumbering{arabic}

%\pagenumbering{arabic} % serve per mettere i numeri romani
\chapter*{\hl{Abstract}} % crea l'introduzione (un capitolo non numerato)

% imposta l'intestazione di pagina
\rhead[\fancyplain{}{\bfseries ABSTRACT}]{\fancyplain{}{\bfseries\thepage}}
\lhead[\fancyplain{}{\bfseries\thepage}]{\fancyplain{}{\bfseries ABSTRACT}}

% aggiunge la voce Introduzione nell'indice
\addcontentsline{toc}{chapter}{Abstract}

In questo lavoro di tesi si vuole presentare un nuovo linguaggio che fonda le sue basi su AgentSpeak e tuProlog permettendo quindi di implementare il modello ad agenti, ereditato da AgentSpeak, su ambienti o piattaforme diverse, grazie alla flessibilità di tuProlog. Verrà preso in esame l'utilizzo del linguaggio in combinazione con il meta-simulatore Alchemist e ne verranno descritti i dettagli implementativi.

% non numera l'ultima pagina sinistra
\clearpage{\pagestyle{empty}\cleardoublepage}

%crea l'indice
\tableofcontents

% imposta l'intestazione di pagina
\rhead[\fancyplain{}{\bfseries\leftmark}]{\fancyplain{}{\bfseries\thepage}}
\lhead[\fancyplain{}{\bfseries\thepage}]{\fancyplain{}{\bfseries INDICE}}

% non numera l'ultima pagina sinistra
\clearpage{\pagestyle{empty}\cleardoublepage}

% crea l'elenco delle figure
\listoffigures

% non numera l'ultima pagina sinistra
\clearpage{\pagestyle{empty}\cleardoublepage}

% crea l'elenco delle tabelle
%\listoftables

% non numera l'ultima pagina sinistra
\clearpage{\pagestyle{empty}\cleardoublepage}

% crea l'elenco dei codici sorgenti
\lstlistoflistings

% non numera l'ultima pagina sinistra
\clearpage{\pagestyle{empty}\cleardoublepage}

% crea il CAPITOLO
\chapter{Introduzione}
% imposta l'intestazione di pagina
\lhead[\fancyplain{}{\bfseries\thepage}]{\fancyplain{}{\bfseries\rightmark}}
In questo lavoro di tesi si vuole realizzare un nuovo linguaggio ad agenti, che si ispira al modello BDI di AgentSpeak, che permetta di essere utilizzato in ambienti o piattaforme differenti, ovvero sia simulati che reali.

\paragraph{Obiettivo del lavoro.}
L'obiettivo del lavoro di tesi è quello utilizzare la definizione di agenti BDI fatta da AgentSpeak per definire un nuovo linguaggio ad agenti al quale, inoltre, si è voluto aggiungere anche una caratteristica di flessibilità. Quest'ultima è stata raggiunta grazie all'utilizzo della libreria tuProlog che ha permesso di definire un linguaggio che possa essere utilizzato da interpreti realizzati su ambienti e piattaforme differenti accomunate dall'utilizzo di questa libreria.

AgentSpeak è un linguaggio orientato agli agenti basato sulla programmazione logica e l'architettura BDI per programmare agenti autonomi. Il modello BDI (Beliefs, Desires, Intentions) implementa gli aspetti principali del ragionamento umano e consente di programmare agenti intelligenti.
tuProlog è una libreria che permette di sfruttare il linguaggio Prolog, incapsulato in un core minimale, all'interno di applicazioni e infrastrutture distribuite.

Si vuole mostrare, inoltre, come è possibile creare un interprete all'interno del meta-simulatore Alchemist, realizzando un'opportuna incarnazione, che sfrutta il modello di agenti BDI definito da AgentSpeak e l'implementazione di Jason, in particolare relativamente al ciclo di ragionamento, e che permette l'esecuzione di agenti, definiti tramite le teorie utilizzando il nuovo linguaggio, in un ambiente simulato.

Alchemist è un simulatore per il calcolo pervasivo, aggregato e naturale, che si basa su un meta-modello flessibile, il quale permette di realizzare implementazioni di modelli completamente diversi tra loro.
Jason è un interprete di una versione estesa di AgentSpeak che implementa un linguaggio e fornisce una piattaforma per sviluppare sistemi multi-agente.

\paragraph{Benefici dell'approccio scelto.}
Il beneficio del linguaggio risiede intrinsecamente nell'architettura del modello ad agenti BDI poichè cerca di esprimere tutto il potenziale del paradigma ad agenti.
In particolare con la definizione di questo nuovo linguaggio si vuole fornire una soluzione per poter sfruttare a pieno l'implementazione dell'agente separando le sue competenze da quelle, invece, puramente demandate all'interprete.
In questo modo all'interno dell'agente saranno definiti solo i meccanismi di reazione a determinati eventi mentre la parte di scheduling e gestione del modello saranno determinati dall'implementazione dell'interprete.
L'agente quindi si comporterà in modo diverso rispetto a come verrà realizzato l'interprete, ovvero in base all'implementazione di:
\begin{itemize}
\item funzioni di prelazione relative a selezione di piani e intenzioni
\item gestione dei messaggi
\item gestione dell'ambiente esterno
\item selezione degli eventi da far gestire all'agente
\end{itemize}

\paragraph{Struttura dell'elaborato.}
In questo capitolo è stato descritto l'obiettivo del lavoro ed i benefici che si vogliono ottenere.
A seguire è accennato il contenuto dei prossimi capitoli presenti in questo lavoro di tesi.

Nel \cref{chap:soa} sono descritti inizialmente i lavori correlati al modello ad agenti, ai sistemi multi-agente e sistemi distribuiti, a cui segue una descrizione del modello ad agenti formalizzato da AgentSpeak e del ciclo di ragionamento utilizzato dall'interprete Jason.
Successivamente sono descritte le caratteristiche di tuProlog, Alchemist e del modello SpatialTuples, il quale definisce un'estensione del modello base di tuple per i sistemi multi-agente.

Nel \cref{chap:agentspeak-2p} è definito il nuovo linguaggio ad agenti, il quale unisce la solidità del modello AgentSpeak alle semantiche operazionali del ciclo di ragionamento Jason, e le relative API messe a disposizione per poter programmare l'operato di un agente.
All'interno del capitolo è descritto come avviene la gestione delle intenzioni nella teoria dell'agente ed inoltre sono presenti alcuni esempi base che mostrano il funzionamento del linguaggio.

Nel \cref{chap:agentspeak-2p-alchemist} è iniziata la descrizione dell'implementazione dell'interprete che, per questo progetto di tesi, è stato realizzato all'interno del simulatore Alchemist.
Inizia dall'analisi del mapping per proseguire con la descrizione della libreria, implementata per gestire le primitive del linguaggio che è stato definito, e di alcune situazioni, come la gestione delle azioni e lo spostamento del nodo, in cui questa è stata utilizzata per la gestire il collegamento tra l'interprete e la teoria.

Nel \cref{chap:impl} si è scesi più nello specifico dell'implementazione dell'interprete descrivendo come sono state create le varie classi, in modo particolare le entità Incarnation, Node e Action.
Successivamente è mostrato un ulteriore dettaglio sull'implementazione di alcune funzionalità peculiari dell'interprete, quali la gestione dei messaggi attraverso l'agente Postman, la referenziazione di oggetti Java nelle teorie tramite tuProlog e il modo in cui sono create le intenzioni.
A seguire sono accennate brevemente le caratteristiche dei tool di sviluppo utilizzati e di come scrivere la configurazione di una simulazione.

Nel \cref{chap:validation} è descritto uno scenario di test noto nei sistemi multi-agente, il problema dei Goldminers, e sono descritte le classi e le teorie implementate per poterlo realizzare.
All'interno del capitolo è presente una sezione per spiegare come è stato possibile tematizzare la simulazione; grazie allo stile creato è stato possibile controllare lo sviluppo dello scenario nel tempo in maniera più semplice.
Nella sezione delle metriche di progetto, sono descritti alcuni indicatori che permettono di analizzare meglio il lavoro svolto.

Infine, nel \cref{chap:conclusions}, sono descritti e analizzati i risultati ottenuti sia attraverso il linguaggio che tramite la definizione dell'interprete.

% crea il CAPITOLO
\chapter{Alchemist}
% imposta l'intestazione di pagina
\lhead[\fancyplain{}{\bfseries\thepage}]{\fancyplain{}{\bfseries\rightmark}}
% mette i numeri arabi
%\pagenumbering{arabic}

Per la realizzazione dell'ambiente per il progetto di tesi è stato scelto il simulatore Alchemist, poichè fornisce già una struttura ed un meta-modello solido e utilizzabile. Nel capitolo viene mostrato Alchemist attraverso le funzionalità e ciò che lo compone. Successivamente è presentata una serie di peculiarità di questo strumento che saranno approfondite più avanti nel documento.

%----------------------------
\section{Descrizione Alchemist}
Alchemist è un simulatore per il calcolo pervasivo, aggregato e ispirato alla natura. Esso fornisce un ambiente di simulazione sul quale è possibile sviluppare nuove incarnazioni, cioè nuove definizioni di modelli implementati su di esso. Ad oggi sono disponibili le funzionalità per:
\begin{itemize}
\item simulare un ambiente bidimensionale;
\item simulare mappe del mondo reale, con supporto alla navigazione e importazione di tracciati in formato gpx;
\item simulare ambienti indoor importando immagini in bianco e nero;
\item eseguire simulazioni biologiche utilizzando reazioni in stile chimico;
\item eseguire programmi Protelis, Scafi, SAPERE (scritti in un linguaggio basato su tuple come LINDA).
\end{itemize}

\subsection{Meta-modello}
Il meta-modello di Alchemist può essere compreso osservando la figura \ref{fig:alchemistModel}.
%crea l'ambiente figura;
\begin{figure}[h] % [h] sta per here, cioè la figura va qui
\begin{center} % centra nel mezzo della pagina la figura
\includegraphics[width=12.5cm]{images/AlchemistModel.png} % inserisce una figura larga 12.5cm
% inserisce la legenda ed etichetta la figura con \label{fig:prima}
\caption[Illustrazione meta-modello di Alchemist]{Illustrazione meta-modello di Alchemist} \label{fig:alchemistModel}
\end{center}
\end{figure}

L'\textbf{\textit{Environment}} è l'astrazione dello spazio ed è anche l'entità più esterna che funge da contenitore per i nodi. Conosce la posizione di ogni nodo nello spazio ed è quindi in grado di fornire la distanza tra due di essi e ne permette inoltre lo spostamento.

\`E detta \textbf{\textit{Linking rule}} una funzione dello stato corrente dell'environemnt che associa ad ogni nodo un \textbf{\textit{Vicinato}}, il quale è un entità composta da un nodo centrale e da un set di nodi vicini.

Un \textbf{\textit{Nodo}} è un contenitore di molecole e reazioni che è posizionato all'interno di un environment.

La \textbf{\textit{Molecola}} è il nome di un dato, paragonabile a quello che rappresenta il nome di una variabile per i linguaggi imperativi.
Il valore da associare ad una molecola è detto \textbf{\textit{Concentrazione}}.

%crea l'ambiente figura;
\begin{figure}[h] % [h] sta per here, cioè la figura va qui
\begin{center} % centra nel mezzo della pagina la figura
\includegraphics[width=14cm]{images/AlchemistReaction.png} % inserisce una figura larga 12.5cm
% inserisce la legenda ed etichetta la figura con \label{fig:prima}
\caption[Illustrazione modello reazione di Alchemist]{Illustrazione modello reazione di Alchemist} \label{fig:alchemistReaction}
\end{center}
\end{figure}

Una \textbf{\textit{Reazione}} è un qualsiasi evento che può cambiare lo stato dell'environment ed è definita tramite una distribuzione temporale, una lista di condizioni e una o più azioni.
\\La frequenza con cui avvengono dipende da:
\begin{itemize}
\item un parametro statico di frequenza;
\item il valore di ogni condizione;
\item un'equazione di frequenza che combina il parametro statico e il valore delle condizioni restituendo la frequenza istantanea;
\item una distribuzione temporale.
\end{itemize}
Ogni nodo contiene un set di reazioni che può essere anche vuoto.

Per comprendere meglio il meccanismo di una reazione si può osservare la figura \ref{fig:alchemistReaction}.

Una \textbf{\textit{Condizione}} è una funzione che prende come input l'environment corrente e restituisce come output un booleano e un numero. Se la condizione non si verifica, le azioni associate a quella reazione non saranno eseguite. In relazione a parametri di configurazione e alla distribuzione temporale, una condizione potrebbe influire sulla velocità della reazione.

La \textbf{\textit{Distribuzione temporale}} indica il numero di eventi, in un dato intervallo di tempo, generati da Alchemist e che innescano la verifica delle condizioni che possono portare alla potenziale esecuzione delle azioni.

Un'\textbf{\textit{Azione}} è la definizione di una serie di operazioni che modellano un cambiamento nel nodo o nell'environment.

In Alchemist un'incarnazione è un'istanza concreta del meta-modello appena descritta e che implementa una serie di componenti base come: la definizione di una molecola e del tipo di dati della concentrazione, un set di condizioni, le azioni e le reazioni. Incarnazioni diverse possono modellare universi completamente differenti.


\section{Aspetti principali in Alchemist}
Alchemist è uno strumento molto esteso e che può offrire tantissime possibilità se lo si conosce e si è in grado di padroneggiarlo.
La conoscenza iniziale di questo ambiente era però praticamente nulla e, quindi è stato necessario impiegare del tempo per riuscire a padroneggiare i meccanismi di base, tra cui l'utilizzo delle classi delle entità del meta-modello e la scrittura della configurazione di una simulazione.

Gli aspetti principali di Alchemist sono la forte adattabilità del meta-modello sul quale è costrutito, il numero di implementazioni o astrazioni base già disponibili e la grande personalizzazione delle simulazioni.
\\
Il meta-modello fornisce una struttura molto solida sulla quale è possibile realizzare ambiti applicativi anche molto diversi tra loro.
\\
Inoltre, Alchemist, come citato poco fa, fornisce già una serie di implementazioni o astrazioni di classi relative a entità del meta-modello che permettono di avviare lo sviluppo di un'incarnazione in modo molto più rapido.
\\
Per quanto riguarda le simulazioni, queste sono realizzate tramite una configurazione che consiste in una mappa definita tramite il linguaggio YAML. La mappa è composta da diverse sezioni, ognuna caratterizzata da una specifica keyword, ed è altamenta configurabile: questo permette all'utente di testare tantissimi aspetti della simulazione.







\chapter{AgentSpeak in tuProlog}
Nel capitolo precedente è stato descritto lo stato attuale di lavori correlati che utilizzano il modello ad agenti BDI per costruirne altri più complessi ed espressivi o implementano linguaggi basati sugli agenti. Inoltre, è stato mostrato lo stato dell'arte delle tecnologie che sono state utilizzate.

In questo lavoro di tesi si è voluto definire un nuovo linguaggio che fosse \textit{`platform indipendent'}, ovvero indipendente dall'ambiente sul quale viene utilizzato: è stato definito al pari di una libreria, senza nessun riferimento all'ambiente. Nei capitoli successivi verrà mostrato come, partendo dal linguaggio, è stata colmata la distanza con l'ambiente scelto.

\section{Definizione linguaggio}\label{sctn:definizioneLinguaggio}
Essendo tuProlog un interprete che opera su piattaforme differenti, si è voluto utilizzarlo nella definizione del linguaggio per permettere di utilizzare quest'ultimo facilmente, potendo sfruttare la libreria tuProlog per colmare la distanza tra l'ambiente e il linguaggio.

Seguendo quella che è la struttura di AgentSpeak, è stato formalizzato questo linguaggio di programmazione ad agenti, e, di seguito, sono mostrate le definizioni.

\smallskip
% Definizione 1
\begin{defn}
Se $b$ è un simbolo di predicato e $t_1, \ldots, t_n$ sono termini, allora $belief(b(t_1, \ldots, t_n))$ è un atomo di belief.
Se $belief(b(t))$ e $belief(c(s))$ sono atomi di belief, allora $belief(b(t)) \land belief(c(s))$ e $\neg belief(b(t))$ sono beliefs.
Un atomo di belief oppure la sua negazione sono riferiti al letterale del belief. Un atomo di belief base sarà chiamato \textit{belief base}.
\end{defn}

\smallskip
% Definizione 2
\begin{defn}\label{defn:goals}
Se $g$ è un simbolo di predicato e $t_1, \ldots, t_n$ sono termini, allora $achievement(g(t_1, \ldots, t_n))$ e $test(g(t_1, \ldots, t_n))$ sono \textit{goals}.
\end{defn}

\smallskip
% Definizione 3
\begin{defn}\label{defn:triggeringEvents}
Se $b(t)$ è un atomo di belief e $g(t)$ un goal, allora $onAddBelief(b(t))$, $onRemoveBelief(b(t))$, $onReceivedMessage(b(t))$, $onResponseMessage(b(t))$, $concurrent(achievement(g(t)))$, $concurrent(test(g(t)))$ sono \textit{eventi di attivazione}.
\end{defn}

\smallskip
% Definizione 4
\begin{defn}
Se $a$ è un simbolo di azione e $t_1, \ldots, t_n$ sono termini del primo ordine, allora $a(t_1, \ldots, t_n)$ è un'azione.
\end{defn}

\smallskip
% Definizione 5
\begin{defn}
%Se $e$ è un \textit{evento di attivazione} e $h_1, \ldots, h_n$ sono goals o azioni, allora $e :- $h_1, \ldots, h_n$ è un piano. L'espressione a sinistra di `:-' è la testa, quella sulla destra il corpo: se quest'ultimo è vuoto viene definito con l'espressione \textit{true}.
Se $e$ è un \textit{evento di attivazione}, $b_1, \ldots, b_m$ sono belief o guardie e $h_1, \ldots, h_n$ sono goals o azioni, allora $`\leftarrow'(e, [b_1, \ldots, b_m], [h_1, \ldots, h_n])$ è un piano.
\end{defn}

\smallskip
% Definizione 6
\begin{defn}\label{defn:intenzione}
Ogni intenzione ha al suo interno uno stack di piani parzialmente istanziati, ovvero dove alcune delle variabili sono state istanziate. Un'intenzione è definita come $intention(i, [p_1, \ldots, p_n])$, dove $i$ è l'identificativo univoco dell'intenzione e $[p_1, \ldots,p_n]$ è lo stack formata da azioni, belief o goal: $p_1$ è la coda e $p_n$ è la testa.
\end{defn}

\smallskip
% Definizione 7
\begin{defn}
Un \textit{agente} è formato da $\langle B,P,I,A,S_O,S_I \rangle$, dove $B$ è una `belief base', $P$ è un set di piani, $I$ è un set di intenzioni, $A$ è un set di azioni. La funzione $S_O$ sceglie un piano dal set di quelli applicabili; la funzione $S_I$ sceglie l'intenzione da eseguire dal set $I$.
\end{defn}

\smallskip
% Definizione 8
\begin{defn}
Dato un evento $\epsilon$ ed un piano $p = `\leftarrow'(e, [b_1, \ldots, b_m], [h_1, \ldots, h_n])$, allora $p$ è rilevante per l'evento $\epsilon$ se e solo se esiste un unificatore $\sigma$ tale per cui $d\sigma = e\sigma$. $\sigma$ è detto \textit{unificatore rilevante} per $\epsilon$.
\end{defn}

\smallskip
% Definizione 9
\begin{defn}
Un piano $p$ è definito da $`\leftarrow'(e, [b_1, \ldots, b_m], [h_1, \ldots, h_n])$ è un \textit{piano applicabile} rispetto ad un evento $\epsilon$ se e solo se esite un identificatore rilevante $\sigma$ per $\epsilon$ e esiste una sostituzione $\theta$ tale che $\forall (b_1, \ldots, b_m) \sigma\theta$ è una conseguenza logica di $B$.
%La composizione $\sigma\theta$ è riferita all'\textit{unificatore applicabile} per l'evento $\epsilon$ e $\theta$ è riferita alla sostituzione della corretta risposta.
\end{defn}

%\bigskip
%TODO
%
%\textbf{selezione intenzione (13,14,15,16)}
%
%\smallskip
%% Definizione 13
%\begin{defn}
%Sia $S_I(I) = i$, dove $i$ è $[p_1 \ddagger \ldots \ddagger f : c_1 \land \ldots \land c_y \leftarrow !g(t);h_2; \ldots; h_n]$. L'intenzione $i$ si dice che è eseguita  se e solo se $\langle +!g(t), i \rangle \in E$.
%\end{defn}
%
%\smallskip
%% Definizione 14
%\begin{defn}
%Sia $S_I(I) = i$, dove $i$ è $[p_1 \ddagger \ldots \ddagger f : c_1 \land \ldots \land c_y \leftarrow ?g(t); h_2; \ldots; h_n]$. L'intenzione $i$ si dice che è eseguita  se e solo se esiste una sostituzione $\theta$ tale che $\forall g(t) \theta$ è una conseguenza logica di B e $i$ è rimpiazzato da $[p_1 \ddagger \ldots \ddagger (f : c_1 \land \ldots \land c_y) \theta \leftarrow h_2 \theta; \ldots; h_n \theta]$.
%\end{defn}
%
%\smallskip
%% Definizione 15
%\begin{defn}
%Sia $S_I(I) = i$, dove $i$ è $[p_1 \ddagger \ldots \ddagger f : c_1 \land \ldots \land c_y \leftarrow a(t); h_2; \ldots; h_n]$. L'intenzione $i$ si dice che è eseguita  se e solo se $a(t) \in A$, e $i$ è rimpiazzato da $[p_1 \ddagger \ldots \ddagger f : c_1 \land \ldots \land c_y \leftarrow h_2; \ldots; h_n]$.
%\end{defn}
%
%\smallskip
%% Definizione 16
%\begin{defn}
%Sia $S_I(I) = i$, dove $i$ è $[p_1 \ddagger \ldots \ddagger p_{z-1} \ddagger g(t) : c_1 \land \ldots \land c_y \leftarrow true]$, dove $p_{z-1}$ è $e : b_1 \land \ldots \land b_x \leftarrow !g(s); h_2; \ldots; h_n$. L'intenzione $i$ si dice che è eseguita  se e solo se esiste una sostituzione $\theta$ tale che $g(t)\theta = g(s)\theta$ e $i$ è rimpiazzato da $[p_1 \ddagger \ldots \ddagger p_{z-1} \ddagger (e : b_1 \land \ldots \land b_x)\theta \leftarrow (h_2; \ldots; h_n) \theta]$.
%\end{defn}

\section{API del linguaggio}
Il linguaggio appena definito è ciò che è messo a disposizione del programmatore dell'agente per descrivere il suo comportamento. Oltre questo, sono state definite altre sintassi che permettano al programmatore di gestire ogni evento o situazione per l'agente.
Qui di seguito sono citate regole, variabili e fatti del linguaggio:
\begin{itemize}
\item $init :- \ldots$
\item $self(A).$
\item $agent$
\item $node$
\item $addBelief(B).$
\item $removeBelief(B).$
\item $onAddBelief(B) :- \ldots$
\item $onRemoveBelief(B) :- \ldots$
\item $onReceivedMessage(S, M) :- \ldots$
\item $achievement(t).$
\item $test(t).$
\item $concurrent(t).$
\item $belief(position(X,Y)).$
\item $belief(distance(A, ND, OD)).$ oppure $belief(distance(A, ND)).$
\end{itemize}

Di seguito vengono analizzate ed esposte.
\\
Per ogni regola è lasciata l'implementazione del corpo al programmatore dell'agente.

La regola $`init'$ è messa a disposizione per permettere di far effettuare una configurazione iniziale dell'agente. Infatti, questa regola verrà invocata solo ed esclusivamente la prima volta che viene attivato l'agente, al posto del ciclo di ragionamento. In questo modo il programmatore dell'agente è in grado di far eseguire all'agente una serie di operazioni iniziali per impostare ad esempio la `belief base' dell'agente.

\medskip
Il fatto $`self(A)'$ permette all'agente di recuperare il suo nome. In questo modo il nome dell'agente può essere recuperato anche all'interno della teoria tuProlog.

\medskip
I due letterali $`agent'$ e $`node'$ sono due variabili di tuProlog alle quali sono collegati gli oggetti dell'agente e del nodo, implementati nell'ambiente su cui si è scelto di utilizzare il linguaggio. Se costruiti correttamente, dalla teoria dell'agente sarà possibile richiamare metodi implementati nella classe corrispondente. La variabile $`agent'$ fa riferimento all'oggetto dell'agente stesso, mentre $`node'$ si riferisce all'oggetto che rappresenta lo spazio sul quale l'agente è inserito. In questo modo possono essere gestite le azioni interne ed esterne dell'agente.

\medskip
Le regole $`addBelief(B)'$ e $`removeBelief(B)'$ sono utilizzabili per aggiungere o rimuovere elementi dalla `belief base'. Il loro utilizzo scatena un evento che va ad invocare $`onAddBelief(B)'$, $`onRemoveBelief(B)'$. Più precisamente $`onAddBelief(B)'$ viene invocato quando viene aggiunto un belief, mentre $`onRemoveBelief(B)'$ è chiamato in seguito alla rimozione di un belief dalla `belief base'. In entrambi i casi la variabile $B$ corrisponde al belief inserito o rimosso.
\\
Diversamente, quando viene letto un messaggio ricevuto da un altro agente (o anche da se stesso), è invocato $`onReceivedMessage(S, M)'$, dove $S$ rappresenta il mittente e $M$ il contenuto del messaggio, che consente all'agente di reagire quando viene letto un messaggio tra quelli presenti nella sua coda di ingresso.

\medskip
Come visto precedentemente nella Definizione \ref{defn:goals}, i letterali $`achievement'$, $`test'$ sono utilizzati per impostare dei goal nell'agente. Ciò che viene scatenato  è l'inserimento della serie di operazioni definita dal goal in testa allo stack dell'intenzione.
In combinazione, i due letterali appena citati possono essere usati in combinazione con $`concurrent'$, mostrato nella Definizione \ref{defn:triggeringEvents}, che permette di inserire le operazioni definite nel goal in una nuova intenzione. In questo modo, la nuova intenzione può essere eseguita in modo concorrente o parallelo rispetto a quella `padre'.

\medskip
Per rendere disponibile al programmatore dell'agente varie possibilità per accedere a informazioni quali la posizione dell'agente e la distanza degli altri agenti rispetto alla propria posizione, sono utilizzati due belief che saranno aggiornati direttamente dall'ambiente sul quale viene utilizzato il linguaggio. La posizione dell'agente viene aggiornata una volta per ogni ciclo di ragionamento e, al termine, sono modificati anche i valori dei belief appena citati. 
\\
Per quanto riguarda la posizione dell'agente, potrà essere invocato $`belief(position(X, Y))'$ dove $X$ è la coordinata relative alle ascisse o longitudine e $Y$ è la coordinata relativa alle ordinate o latitudine.
\\
La distanza da altri agenti può essere molto utile per far scegliere all'agente di effettuare o meno una certa azione. Se nella lista del vicinato entra un nuovo agente viene inserito il belief $`belief(distance(A, ND))'$, dove $A$ è il nome dell'agente nel vicinato e $ND$ è la distanza che li separa. Se, invece, un agente era già nella lista del vicinato e vi rimane, allora viene inserito il belief $`belief(distance(A, ND, OD))'$, dove $A$ è il nome dell'agente nel vicinato, $ND$ è la nuova distanza che li separa e $OD$ è la distanza che li divideva precedentemente.

\subsection{Gestione intenzioni}
Le intenzioni sono la modalità con cui l'agente opera le sue azioni. Come descritto in precedenza, nel ciclo di ragionamento alla sezione \ref{ssctn:cicloRagionamentoAgentSpeak}, l'agente esegue una serie di passi che portano all'esecuzione di un'azione.
Qui di seguito è descritto come avviene il ciclo di ragionamento utilizzando questo linguaggio. La spiegazione terrà conto solamente degli aspetti relativi alla parte tuProlog e quindi sarà incompleta fino al raggiungimento della sezione \ref{sctn:interpreteLinguaggio}. Le funzioni di selezione per i piani applicabili e le intenzioni non sono trattate in questa parte, poichè sono relative all'implementazione dell'interprete.

L'agente lato tuProlog definisce il suo comportamento tramite una serie di regole e fatti che gli permettono di reagire ad eventi sia esterni che interni. Una percezione dell'ambiente può essere scatenata ad esempio da uno spostamento o una modifica della `belief base': quando questo avviene l'ambiente sul quale è utilizzato il linguaggio invoca una delle regole che, se implementata correttamente nella teoria dell'agente, consente all'agente di reagire all'evento. Un altro tipo di input che può ricevere l'agente è la ricezione di un messaggio. In tuProlog l'agente può reagire alla lettura del contenuto del messaggio poichè l'implementazione e la gestione delle code e la selezione dei messaggi viene fatta dall'interprete.

La frequenza dell'esecuzione del ciclo di ragionamento dipende dall'ambiente sul quale viene utilizzato il linguaggio. All'interno del ciclo, una volta selezionato il piano applicabile per l'evento avvenuto, l'interprete invoca delle regole per ottenere la lista delle operazioni presenti nel corpo del piano (o regola) per poterle inserire nell'intenzione. Le regole per recuperare la lista eseguono per ogni elemento del corpo una lettura e un inserimento all'interno di una lista, la quale poi viene restituita.

La creazione dell'intenzione viene fatta dall'interprete ma salvata come fatto nella teoria dell'agente. Come descritto nella definizione \ref{defn:intenzione}, l'intenzione $intention(id,[op_1, \ldots, op_n])$ è composta da un identificativo univoco $id$ e da una lista di operazioni $[op_1, \ldots, op_n]$. All'interno della teoria dell'agente possono essere presenti più intenzioni contemporaneamente ma ad ogni ciclo di ragionamento solo una verrà selezionata per l'esecuzione. Come detto precedentemente, anche la selezione dell'intenzione è gestita dall'interprete ma, lato tuProlog viene gestita l'esecuzione. Infatti, l'interprete si limita a invocare la regola $execute(I)$ all'interno della quale viene gestita l'esecuzione della prima operazione sullo stack dell'intenzione con identificativo $I$, ovvero quella che è stata precedentemente selezionata.
\\
La regola $execute$ si occupa di recuperare l'intenzione riferita all'identificativo passato e quindi prendere la testa dello stack delle operazioni. Quest'ultima viene valutata ed in base alla sua natura vengono eseguite azioni diverse:
\begin{itemize}
\item le azioni vengono eseguite direttamente;
\item i goal vengono recuperati lo stack di operazioni collegate viene successivamente aggiunto in testa all'intenzione di cui faceva parte il goal;
\item i goal espressi all'interno di $concurrent$ creano una nuova intenzione che potrà essere eseguita in parallelo rispetto a quella da cui ha avuto origine la chiamata al goal.
\end{itemize}


\subsection{Estensione Spatial Tuples}
Il linguaggio appena descritto è stato esteso per permettere di utilizzare il modello Spatial Tuples. Sono state quindi inserite le seguenti regole:
\begin{itemize}
\item $writeTuple(T).$
\item $readTuple(TT).$
\item $takeTuple(TT).$
\item $onResponseMessage(M) :- \ldots$
\end{itemize}
Le regole dell'elenco sono tutte riferite all'inserimento, all'interno del linguaggio, del modello di coordinazione LINDA e più precisamente del modello Spatial Tuples.
Con questa estensione, viene data la possibilità agli agenti di poter inserire e recuperare informazioni posizionate nello spazio. Le regole messe a disposizione mappano le primitive dei modelli che vogliono implementare `\textit{in}', `\textit{rd}', `\textit{out}' rispettivamente in $`writeTuple(T)'$, $`readTuple(TT)'$, $`takeTuple(TT)'$ dove $T$ è la tupla da inserire e $TT$ è il template da ricercare.
\\
Utilizzando $`writeTuple(T)'$ il programmatore è in grado di inserire informazioni posizionate nello spazio degli agenti e con le quali gli stessi agenti possono interagire. Per leggere le informazioni si possono utilizzare due diverse modalità: $`readTuple(TT)'$ e $`takeTuple(TT)'$. Nel primo caso viene utilizzato il template passato per confrontarlo con le tuple nell'intorno dell'agente e se ci sono risultati che combaciano con il template allora uno di questi viene restituito. Per quanto riguarda invece $`takeTuple(TT)'$, si comporta ugualmente per quanto riguarda la ricerca della tupla con il template ma poi, una volta trovati i risultati ne sceglie uno e prima di restituirlo lo elimina dallo spazio di tuple in cui era presente.
\\
Entrambe le due modalità di getter seguono la semantica standard dei modelli basati su tuple, e quindi sono:
\begin{itemize}
\item sospensive: se non ci sono tuple che si abbinano al template l'operazione è bloccata finchè non viene trovata una tupla;
\item non deterministiche: se ci sono più tuple che si abbinano al template una è scelta in modo non deterministico.
\end{itemize}
Per dare la possibilità di gestire la risposta e gestire la tupla restituita è stata introdotta $`onResponseMessage(M)'$ che viene invocata ogni qualvolta che una tupla viene restituita dallo spazio di tuple. Il contenuto $M$ è la tupla incapsulata in un belief in modo che si possano gestire tuple provenienti da diversi spazi di tuple e con contenuti differenti.

\section{Esempi linguaggio}
In questa sezione verranno mostrate alcuni casi d'uso del linguaggio appena descritto. Nello specifico verrà mostrato un primo scenario dove sono stati configurati gli agenti per realizzare un semplice scambio di messaggi (o Ping Pong). Nel secondo esempio, invece, viene illustrato come poter utilizzare l'estensione Spatial Tuples supportata dal linguaggio.

\subsubsection{Ping Pong}
In questo primo esempio è presentato il problema del Ping Pong. In questo esempio sono definiti due agenti, Ping e Pong, ognuno dei quali risponde ad un messaggio ricevuto. L'agente Ping, alla ricezione del messaggio `\textit{pong}' da parte dell'agente Pong risponderà con un messaggio `\textit{ping}'. Viceversa, l'agente Pong, alla ricezione del messaggio `\textit{ping}' da parte dell'agente Ping risponderà con un messaggio `\textit{pong}'.

Per far iniziare lo scambio di messaggi è stato utilizzato `init' per impostare all'interno di uno dei due agenti, nello specifico l'agente Ping, un'intenzione iniziale. In questo modo, al primo ciclo di ragionamento, l'agente eseguirà l'intenzione e invierà il primo messaggio.
\lstset{
  %numberstyle=\footnotesize\color{black},
  basicstyle=\ttfamily,
  %breakatwhitespace=false,
  %breaklines=true,
  captionpos=b,
  %keepspaces=true,
  %numbers=left,
  %numbersep=0pt,
  %showspaces=false,
  %showstringspaces=false,
  %showtabs=false,
  frame=tb,
  %commentstyle=\color{black},
  %keywordstyle=\color{black},
  %stringstyle=\color{black}
  %label=incarnationYAML,
  %caption={First verbatim}
  %language=Java
  %escapeinside={(*@}{@*)}
}
\medskip
\begin{lstlisting}[firstnumber=1,label={lst:PingAgent},caption={Agente Ping}]
init :-
  addBelief(intention(0,[iSend('pong_agent','ping')])),
  agent <- insertIntention(0).

onReceivedMessage(S,pong) :-
  iSend(S, ping).
\end{lstlisting}

In entrambe le teorie dei due agenti è stata richiamata $`iSend(S, M)'$, dove $S$ è il destinatario e $M$ è il messaggio, che è un'azione interna dichiarata e gestita nell'ambiente sul quale è utilizzato il linguaggio. Nel Codice sorgente \ref{lst:PingAgent} viene inviato all'agente Pong il messaggio `ping', mentre nel Codice sorgente \ref{lst:PongAgent} il messaggio inviato all'agente Ping è `pong'.

\medskip
\begin{lstlisting}[firstnumber=1,label={lst:PongAgent},caption={Agente Pong}]
onReceivedMessage(S,ping) :-
  iSend(S, pong).
\end{lstlisting}

\subsubsection{Message passing through Spatial Tuples}
In questo esempio viene mostrato come possono essere utilizzate le primitive del modello Spatial Tuples incorporate nel linguaggio descritto in precedenza. Nello specifico viene mostrato come tre agenti (Alice, Bob e Carl) comunicano tra loro inserendo messaggi negli spazi di tuple a loro vicini, usandoli come `lavagna'.
L'agente Alice nel suo ciclo di configurazione, esegue due scritture sulla `lavagna' (spazio di tuple) inserendo messaggi per Bob e Carl e successivamente effettua altre due richieste allo spazio di tuple richiedendo due messaggi a lei destinati senza conoscerne il contenuto. Una volta ricevuti i messaggi non fa niente.

\medskip
\begin{lstlisting}[firstnumber=1,label={lst:Alice},caption={Alice}]
init :-
  writeTuple(blackboard,msg(bob,hello)),
  writeTuple(blackboard,msg(carl,hello)),
  takeTuple(blackboard,msg(alice,X)),
  takeTuple(blackboard,msg(alice,X)).

onResponseMessage(msg(X,Y)) :- true.
\end{lstlisting}

L'agente Bob, nel suo ciclo di configurazione, effettua una richiesta allo spazio di tuple per ricevere messaggi a lui destinati. Inoltre, nella sua teoria, è definito un comportamento in caso di ricezione del messaggio: manda ad Alice lo stesso messaggio che ha ricevuto.
\medskip
\begin{lstlisting}[firstnumber=1,label={lst:Bob},caption={Bob}]
init :-
  takeTuple(blackboard,msg(bob,X)).

onResponseMessage(msg(bob,X)) :-
  writeTuple(blackboard,msg(alice,X)).
\end{lstlisting}

Come Bob, l'agente Carl esegue lo stesso comportamento di Bob.
\medskip
\begin{lstlisting}[firstnumber=1,label={lst:Carl},caption={Carl}]
init :-
  takeTuple(blackboard,msg(carl,X)).

onResponseMessage(msg(carl,X)) :-
  writeTuple(blackboard,msg(alice,X)).
\end{lstlisting}


% crea il CAPITOLO
\chapter{Agent speak}
% imposta l'intestazione di pagina
\lhead[\fancyplain{}{\bfseries\thepage}]{\fancyplain{}{\bfseries\rightmark}}
% mette i numeri arabi
%\pagenumbering{arabic}

AgentSpeak è un linguaggio orientato agli agenti basato sulla programmazione logica e sul modello BDI (Belief-Desires-Intention). All'interno del capitolo viene descritto cos'è il modello BDI e come funziona AgentSpeak, ed in particolare il ciclo di ragionamento, esemplificato da quello di Jason.

%----------------------------
\section{Agenti BDI con AgentSpeak}
Il modello BDI consente di rappresentare le caratteristiche e le modalità di raggiungimento di un obiettivo secondo il paradigma ad agenti. Gli agenti BDI forniscono un meccanismo per separare le attività di selezione di un piano, fra quelli presenti nella sua teoria, dall'esecuzione del piano attivo, permettendo di bilanciare il tempo speso nella scelta del piano e quello per eseguirlo.

I \textbf{beliefs} sono quindi informazioni dello stato dell'agente, ovvero ciò che l'agente sa del mondo \cite{AgentSpeakInJason} il suo insieme è chiamato `belief base' o `belief set'.

I \textbf{desires} rappresentano tutti i possibili piani che un agente potrebbe eseguire \cite{AgentSpeakInJason}. Rappresentano ciò che l'agente vorrebbe realizzare o portare a termine: i \textit{goals} sono desideri che l'agente persegue attivamente ed è quindi bene che tra loro siano coerenti, cosa che non è obbligatoria per quanto riguarda il resto dei desideri.

Le \textbf{intentions} identificano i piani a cui l'agente ha deciso di lavorare o a cui sta già lavorando e a loro volta possono contenere altri piani \cite{AgentSpeakInJason}.

Gli \textbf{eventi} innescano le attività reattive, ovvero la caratteristica di proattività degli agenti, come ad esempio l'aggiornamento dei beliefs, l'invocazione di piani o la modifica dei goals.

\section{Definizione AgentSpeak}
\textbf{AgentSpeak} è un linguaggio di programmazione basato su un linguaggio del primo ordine con eventi e azioni \cite{AgentSpeak(L)}. Il comportamento degli agenti è dettato da quanto definito nel programma scritto in AgentSpeak. I beliefs correnti di un agente sono relativi al suo stato attuale, all'enviroment e agli altri agenti. Gli stati che un agente vuole determinare sulla base dei suoi stimoli esterni e interni sono i desideri \cite{AgentSpeak(L)}. L'adozione di programmi per soddisfare tali stimoli è detta intenzioni.

\subsection{Composizione}
Un'agente in AgentSpeak è formato da una `belief base' e da una serie di piani opportunamente programmati.
La `belief base' è il contenitore dello stato dell'agente, dove sono presenti tutte le informazioni che esso ha in riferimento a se stesso e all'ambiente. Questo set di belief può essere modificato in modo continuo dalle azioni scatenate nel ciclo di ragionamento.
I piani sono sequenze di azioni o goal che permettono all'agente di reagire a situazioni che avvengono nell'ambiente o internamente.
\\
Qui di seguito viene descritto quali sono le fasi del ciclo di ragionamento dell'agente e in che modo viene definito il suo operato. La sequenza di esecuzione di questa iterazione è fondamentale per la realizzazione di un agente e quindi importante comprenderla per realizzare un'architettura appropriata.

\subsection{Ciclo di ragionamento}\label{ssctn:cicloRagionamentoAgentSpeak}
Il ciclo di ragionamento è il modo in cui l'agente prende le sue decisioni e mette in pratica le azioni. Esso è composto di otto fasi: le prime tre sono quelle che riguardano l'aggiornamento dei belief relativi al mondo e agli altri agenti, mentre altre descrivono la selezione di un evento che permette l'esecuzione di un'intenzione dell'agente.

%/---AGGIORNAMENTO BELIEF BASE---/
\subsubsection{a. Percezione ambiente}
La percezione effettuata dall'agente all'interno del ciclo di ragionamento è utilizzata per poter aggiornare il suo stato. L'agente interroga dei componenti capaci di rilevare i cambiamenti nell'ambiente \cite{AgentSpeakInJason} e di emettere dati consultabili utilizzando opportune interfacce.

\subsubsection{b. Aggiornamento beliefs}
Ottenuta la lista delle percezioni è necessario aggiornare la `belief base'. Ogni percezione non ancora presente nel set viene aggiunta e al contrario quelle presenti nel set e che non sono nella lista delle percezioni vengono rimosse \cite{AgentSpeakInJason}.
Ogni cambiamento effettato nella `belief base' produce un evento: quelli generati da percezioni dell'ambiente sono detti eventi esterni; quelli interni, rispetto agli altri, hanno associata un'intenzione.

\subsubsection{c. Ricezione e selezione messaggi}
L'altra sorgente di informazioni per un agente sono gli altri agenti presenti nel sistema. L'interprete controlla i messaggi diretti all'agente e li rende a lui disponibili \cite{AgentSpeakInJason}: ad ogni iterazione del ciclo può essere processato solo un messaggio. Inoltre,può essere assegnata una priotità ai messaggi in coda definendo una funzione di prelazione per l'agente.
\\
Prima di essere processati i messaggi passano all'interno di una funzione di selezione che definisce quali messaggi possano essere accettati dall'agente \cite{AgentSpeakInJason}. Questa funzione può essere implementata ad esempio per far ricevere solo i messaggi di un certo agente.

%/---SELEZIONE EVENTO E ESECUZIONE INTENZIONE---/
\subsubsection{d. Selezione evento}
Gli eventi rappresentano la percezione del cambiamento nell'ambiente o dello stato interno dell'agente \cite{AgentSpeakInJason}, come il goal. Ci possono essere vari eventi in attesa ma in ogni ciclo di ragionamento può esserne gestito uno solo, il quale viene scelto dalla funzione di selezione degli eventi che ne seleziona uno dalla lista di quelli in attesa. Se la lista di eventi fosse vuota si passa direttamente alla penultima fase del ciclo di ragionamento \cite{AgentSpeakInJason}, ovvero la selezione di un'intenzione.

\subsubsection{e. Recupero piani rilevanti}
Una volta selezionato l'evento è necessario trovare un piano che permetta all'agente di agire per gestirlo. Per fare ciò viene recuperata dalla `Plan Library' la lista dei piani rilevanti, verificando quali possano essere unificati con l'evento selezionato \cite{AgentSpeakInJason}. L'unificazione è il confronto relativo a predicati e termini. Al termine di questa fase si otterrà un set di piani rilevanti per l'evento selezionato che verrà raffinato successivamente.

\subsubsection{f. Selezione piano appplicabile}
Ogni piano ha un contesto che definisce con quali informazioni dell'agente può essere usato.
Per piano applicabile si intendeno quelli che, in relazione allo stato dell'agente, possono avere una possibilità di successo. Viene quindi controllato che il contesto sia una conseguenza logica della `belief base' dell'agente \cite{AgentSpeakInJason}. Vi possono anche essere più piani in grado di gestire un evento ma l'agente deve selezionarne uno solo ed impegnarsi ad eseguirlo.
\\
La selezione viene fatta tramite un'apposita funzione che inoltre tiene conto dell'ordinamento dei piani in base alla loro posizione nel codice sorgente oppure dell'ordine di inserimento. Quando un piano è scelto, viene creata un'istanza di quel piano che viene inserita nel set delle intenzioni \cite{AgentSpeakInJason}: sarà l'istanza ad essere manipolata dall'interprete e non il piano nella libreria.

Ci sono due possibili modalità per la creazione di un'intenzione e dipende dal fatto che l'evento selezionato sia esterno o interno \cite{AgentSpeakInJason}. Nel primo caso viene semplicemente creata l'intenzione, altrimenti viene inserita un'altra intenzione in testa a quella che ha generato l'evento, poichè è necessario eseguire fino al completamento un piano per raggiungere tale goal.

\subsubsection{g. Selezione intenzione}
A questo punto, se erano presenti eventi da gestire, è stata aggiunta un'altra intenzione nello stack. Un agente ha tipicamente più di un'intenzione nel set delle intenzioni che potrebbe essere eseguita, ognuna delle quali rappresenta un diverso punto di attenzione \cite{AgentSpeakInJason}. Ad ogni ciclo di ragionamento avviene l'esecuzione di una sola intenzione, la cui scelta è importante per come l'agente opererà nell'ambiente.

\subsubsection{h. Esecuzione intenzione}
L'intenzione, scelta nello step precedente, non è altro che il corpo di un piano formato da una sequenza di istruzioni, ognuna delle quali, una volta eseguita, viene rimossa dall'istanza del piano. Terminata l'esecuzione un'intenzione, quest'ultima viene restituita al set delle intenzioni a meno che non debba aspettare un messaggio o un feedback dell'azione \cite{AgentSpeakInJason}: in questo caso viene memorizzata in una struttura e restituita una volta ricevuta la risposta.
Se un'intenzione è sospesa non può essere selezionata per l'esecuzione nel ciclo di ragionamento.

\subsubsection{Scambio di messaggi}
Lo scambio di messaggi è la comunicazione standard che avviene tra agenti per comunicare tra loro e operare in base al contenuto ricevuto.
La comunicazione definita da AgentSpeak utilizza tre parti. La prima è la coda dei messaggi in input, ovvero una lista contenente tutti i messaggi che il sistema o interprete riceve e che sono destinati all'agente. La seconda è la coda dei messaggi di output che si allunga ogni volta che l'agente vuole inviare un messaggio ad un altro agente. L'ultima è una struttura all'interno della quale vengono memorizzate le intenzioni che sono sospese dall'esecuzione poichè aspettano una risposta dal canale di comunicazione dei messaggi.
\\
L'interprete è il mezzo per il quale i messaggi trasmessi. Esso infatti ha il compito di recuperare tutti i messaggi nella coda in uscita di ogni agente e successivamente recapitarli. Per la consegna viene recuperato l'agente destinatario di ogni messaggio e poi quest'ultimo viene posizionato nella coda di quelli in input dell'agente, in modo tale che possa recuperarne il contenuto al prossimo ciclo di ragionamento.


%----------------------------


%-----------------------------------------
%-----------------------------------------

% imposta l'intestazione di pagina
%\renewcommand{\chaptermark}[1]{\markright{\thechapter \ #1}{}}
%\lhead[\fancyplain{}{\bfseries\thepage}]{\fancyplain{}{\bfseries\rightmark}}
%\appendix % imposta le appendici

% crea l'appendice
%\chapter{Prima Appendice}
% \verb"" è equivalente all'ambiente verbatim, ma si utilizza all'interno di un discorso.
%\verb"\clearpage{\pagestyle{empty}\cleardoublepage}"

% imposta l'intestazione di pagina
%\rhead[\fancyplain{}{\bfseries \thechapter \:Prima Appendice}]
%{\fancyplain{}{\bfseries\thepage}}

% crea l'appendice
%\chapter{Seconda Appendice}

% imposta l'intestazione di pagina
%\rhead[\fancyplain{}{\bfseries \thechapter \:Seconda Appendice}]
%{\fancyplain{}{\bfseries\thepage}}



\begin{thebibliography}{90} % crea l'ambiente bibliografia
\rhead[\fancyplain{}{\bfseries \leftmark}]{\fancyplain{}{\bfseries \thepage}}

% aggiunge la voce Bibliografia nell'indice
\addcontentsline{toc}{chapter}{Bibliografia}

% provare anche questo comando:
%\addcontentsline{toc}{chapter}{\numberline{}{Bibliografia}}
\bibitem{AgentSpeak(L)} AgentSpeak(L): BDI Agents speak out in a logical computable language, Anand S. Rao, 1996
\bibitem{SpatialTuples} Spatial Tuples: Augmenting reality with tuples, Ricci et al., 2017
\bibitem{AgentSpeakInJason} Programming multi-agent systems in AgentSpeak using Jason, Rafael H. Bordini, Jomi Fred Hubner, Michael Wooldridge, 2007

\end{thebibliography}

%% Preambolo per sitografia
\makeatletter
\let\@orig@endthebibliography\endthebibliography
\renewcommand\endthebibliography{%
  \xdef\@kept@last@number{\the\c@enumiv}%
  \@orig@endthebibliography}

\newenvironment{thesitography}[1]
  {\def\bibname{Sitografia}%
   \thebibliography{#1}%
}
  {\@orig@endthebibliography}
\makeatother
%% Preambolo per sitografia

{\fancyplain{}{\bfseries\thepage}}
\begin{thesitography}{90} % crea l'ambiente bibliografia
\rhead[\fancyplain{}{\bfseries \leftmark}]{\fancyplain{}{\bfseries \thepage}}

% aggiunge la voce Bibliografia nell'indice
\addcontentsline{toc}{chapter}{Sitografia}

% provare anche questo comando:
%\addcontentsline{toc}{chapter}{\numberline{}{Bibliografia}}
\bibitem {JADE} https://jade.tilab.com/
\bibitem {Spade} https://pypi.org/project/spade/
\bibitem {SARL} http://www.sarl.io/
\bibitem {JADEX} https://www.activecomponents.org/\#/project/news
\bibitem {Astra} http://astralanguage.com/wordpress/
\bibitem {Jason} http://jason.sourceforge.net/wp/
\bibitem {tuProlog} http://apice.unibo.it/xwiki/bin/view/Tuprolog/
\bibitem {Alchemist} http://alchemistsimulator.github.io
\end{thesitography}

% non numera l'ultima pagina sinistra
%\clearpage{\pagestyle{empty}\cleardoublepage}
%\chapter{Ringraziamenti}
%\thispagestyle{empty}
%Contenuto ringraziamenti
\end{document}
